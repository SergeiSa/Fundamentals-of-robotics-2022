\documentclass{beamer}

\pdfmapfile{+sansmathaccent.map}


\mode<presentation>
{
  \usetheme{Warsaw} % or try Darmstadt, Madrid, Warsaw, Rochester, CambridgeUS, ...
  \usecolortheme{seahorse} % or try seahorse, beaver, crane, wolverine, ...
  \usefonttheme{serif}  % or try serif, structurebold, ...
  \setbeamertemplate{navigation symbols}{}
  \setbeamertemplate{caption}[numbered]
} 


%%%%%%%%%%%%%%%%%%%%%%%%%%%%
% itemize settings

\definecolor{mypink}{RGB}{255, 30, 80}
\definecolor{mydarkblue}{RGB}{60, 160, 255}
\definecolor{mydarkred}{RGB}{160, 30, 30}
\definecolor{mylightred}{RGB}{255, 150, 150}
\definecolor{myred}{RGB}{200, 110, 110}
\definecolor{myblackblue}{RGB}{40, 40, 120}
\definecolor{myblackred}{RGB}{120, 40, 40}
\definecolor{myblue}{RGB}{240, 240, 255}
\definecolor{mygreen}{RGB}{0, 200, 0}
\definecolor{mygreen2}{RGB}{205, 255, 200}
\definecolor{mygray}{gray}{0.8}
% \definecolor{mydarkgray}{gray}{0.4}
\definecolor{mydarkgray}{RGB}{80, 80, 160}

\setbeamertemplate{itemize items}[default]

\setbeamertemplate{itemize item}{\color{myblackred}$\blacksquare$}
\setbeamertemplate{itemize subitem}{\color{mydarkred}$\blacktriangleright$}
\setbeamertemplate{itemize subsubitem}{\color{mygray}$\blacksquare$}

\setbeamercolor{palette quaternary}{fg=white,bg=myred} %mydarkgray
\setbeamercolor{titlelike}{parent=palette quaternary}

\setbeamercolor{palette quaternary2}{fg=white,bg=mydarkred}%black myblue
\setbeamercolor{frametitle}{parent=palette quaternary2}

\setbeamerfont{frametitle}{size=\Large,series=\scshape}
\setbeamerfont{framesubtitle}{size=\normalsize,series=\upshape}





%%%%%%%%%%%%%%%%%%%%%%%%%%%%
% block settings

\setbeamercolor{block title}{bg=red!30,fg=black}

\setbeamercolor*{block title example}{bg=mygreen!40!white,fg=black}

\setbeamercolor*{block body example}{fg= black, bg= mygreen2}


%%%%%%%%%%%%%%%%%%%%%%%%%%%%
% URL settings
\hypersetup{
    colorlinks=true,
    linkcolor=blue,
    filecolor=blue,      
    urlcolor=blue,
}

%%%%%%%%%%%%%%%%%%%%%%%%%%

\renewcommand{\familydefault}{\rmdefault}

\usepackage{amsmath}
\usepackage{mathtools}
\usepackage{mathrsfs}


\usepackage{subcaption}

\usepackage{qrcode}

\DeclareMathOperator*{\argmin}{arg\,min}
\newcommand{\bo}[1] {\mathbf{#1}}

\newcommand{\dx}[1] {\dot{\mathbf{#1}}}
\newcommand{\ma}[4] {\begin{bmatrix}
    #1 & #2 \\ #3 & #4
    \end{bmatrix}}
\newcommand{\myvec}[2] {\begin{bmatrix}
    #1 \\ #2
    \end{bmatrix}}
\newcommand{\myvecT}[2] {\begin{bmatrix}
    #1 & #2
    \end{bmatrix}}
 
 \newcommand{\R}{\mathbb{R}} 
 \newcommand{\T}{^\top}     
    

\newcommand{\mydate}{Fall 2022}
\newcommand{\mygit}{\textcolor{blue}{\href{https://github.com/SergeiSa/Fundamentals-of-robotics-2022}{github.com/SergeiSa/Fundamentals-of-robotics-2022}}}


\newcommand{\bref}[2] {\textcolor{blue}{\href{#1}{#2}}}




%%%%%%%%%%%%%%%%%%%%%%%%%%%%
% code settings

\usepackage{listings}
\usepackage{color}
% \definecolor{mygreen}{rgb}{0,0.6,0}
% \definecolor{mygray}{rgb}{0.5,0.5,0.5}
\definecolor{mymauve}{rgb}{0.58,0,0.82}
\lstset{ 
  backgroundcolor=\color{white},   % choose the background color; you must add \usepackage{color} or \usepackage{xcolor}; should come as last argument
  basicstyle=\footnotesize,        % the size of the fonts that are used for the code
  breakatwhitespace=false,         % sets if automatic breaks should only happen at whitespace
  breaklines=true,                 % sets automatic line breaking
  captionpos=b,                    % sets the caption-position to bottom
  commentstyle=\color{mygreen},    % comment style
  deletekeywords={...},            % if you want to delete keywords from the given language
  escapeinside={\%*}{*)},          % if you want to add LaTeX within your code
  extendedchars=true,              % lets you use non-ASCII characters; for 8-bits encodings only, does not work with UTF-8
  firstnumber=0000,                % start line enumeration with line 0000
  frame=single,	                   % adds a frame around the code
  keepspaces=true,                 % keeps spaces in text, useful for keeping indentation of code (possibly needs columns=flexible)
  keywordstyle=\color{blue},       % keyword style
  language=Octave,                 % the language of the code
  morekeywords={*,...},            % if you want to add more keywords to the set
  numbers=left,                    % where to put the line-numbers; possible values are (none, left, right)
  numbersep=5pt,                   % how far the line-numbers are from the code
  numberstyle=\tiny\color{mygray}, % the style that is used for the line-numbers
  rulecolor=\color{black},         % if not set, the frame-color may be changed on line-breaks within not-black text (e.g. comments (green here))
  showspaces=false,                % show spaces everywhere adding particular underscores; it overrides 'showstringspaces'
  showstringspaces=false,          % underline spaces within strings only
  showtabs=false,                  % show tabs within strings adding particular underscores
  stepnumber=2,                    % the step between two line-numbers. If it's 1, each line will be numbered
  stringstyle=\color{mymauve},     % string literal style
  tabsize=2,	                   % sets default tabsize to 2 spaces
  title=\lstname                   % show the filename of files included with \lstinputlisting; also try caption instead of title
}


%%%%%%%%%%%%%%%%%%%%%%%%%%%%
% URL settings
\hypersetup{
    colorlinks=false,
    linkcolor=blue,
    filecolor=blue,      
    urlcolor=blue,
}

%%%%%%%%%%%%%%%%%%%%%%%%%%

%%%%%%%%%%%%%%%%%%%%%%%%%%%%
% tikz settings

\usepackage{tikz}
\tikzset{every picture/.style={line width=0.75pt}}

\newcommand{\dq} {\dot{\mathbf{q}}}


\title{Trajectory Optimization}
\subtitle{Fundamentals of Robotics, Lecture 11}
\author{by Sergei Savin}
\centering
\date{\mydate}



\begin{document}
\maketitle





\begin{frame}{State-space}
	% \framesubtitle{Parameter estimation}
	\begin{flushleft}
		
		We can define state-space coordinates $\bo{x}$ for a mechanical system $\bo{H} \ddot{\bo{q}} + \bo{C} \dot{\bo{q}} + \bo{g} = \tau$ as follows:
		
		\begin{equation}
			\bo{x} = \begin{bmatrix}
				\bo{q} \\
				\dot{\bo{q}}
			\end{bmatrix}
		\end{equation}
	
	Defining $\bo{u} = \tau$,  $\bo{S}_q = [\bo{I} \ \ \bo{0}]$ and $\bo{S}_v = [\bo{0} \ \ \bo{I}]$ we get $\bo{q} = \bo{S}_q \bo{x}$ and $\dot{\bo{q}} = \bo{S}_v \bo{x}$. The state-space dynamics becomes:
	
	
		\begin{equation}
				\dot{\bo{x}} = 
				\bo{f}(\bo{x}, \bo{u})
				=
		   \begin{bmatrix}
				\bo{S}_v \bo{x} \\
				\bo{H}^{-1} (\bo{u} - \bo{C} \bo{S}_v \bo{x} - \bo{g})
			\end{bmatrix}
		\end{equation}
	
	\end{flushleft}
\end{frame}





\begin{frame}{Trajectory planning problem}
	% \framesubtitle{Parameter estimation}
	\begin{flushleft}
		
		We know how to control a nonlinear system $\dot{\bo{x}} = 
		\bo{f}(\bo{x}, \bo{u})$ obtained from the manipulator equations, as long as we want point-to-point or trajectory following control, and the trajectory is provided. But how do we find a feasible trajectory?
		
		\bigskip
		
		Here are aspects of a typical trajectory problems:
		
		\begin{enumerate}
			\item Drive the system to the desired state $\bo{x}^*(t_f)$ by the time $t_f$.
			
			\item Respect torque limits: $|| u_i || \leq u_{max}$.
			
			\item Respect kinematic constraints: $\dot{ q }_{min} \leq \dot{ q }_i \leq \dot{ q }_{max}$.
			
			\item Respect joint limits: $q_{i, min} \leq q_i \leq q_{i, max}$.
			
		\end{enumerate}
		
	\end{flushleft}
\end{frame}





\begin{frame}{Trajectory planning as optimization}
	% \framesubtitle{Parameter estimation}
	\begin{flushleft}
		
		We can formulate the trajectory planning as an \emph{optimal control problem} (OCP):
		
		\begin{equation}
			\begin{aligned}
				& \underset{\bo{x}(t), \bo{u}(t)}{\text{minimize}}
				& & \int\limits_{t_0}^{t_f} l \left( \bo{x}(t), \bo{u}(t) \right) dt + l_f(\bo{x}(t_f)), \\
				& \text{subject to:}
				& & \dot{\bo{x}} = 
				\bo{f}(\bo{x}, \bo{u}) \\
				& & & \bo{x}_{min} \leq \bo{x}(t) \leq \bo{x}_{max} \\
				& & & \bo{u}_{min} \leq \bo{u}(t) \leq \bo{u}_{max}
			\end{aligned}
		\end{equation}
	
	This is an optimization problem with continuous variables and there are no solvers that can solve it (in a general case). So, our method is to replace the continuous time variables with a finite number of parameters. This is called \emph{transcription}.
		
	\end{flushleft}
\end{frame}



\begin{frame}{Types of transcription}
	% \framesubtitle{Parameter estimation}
	\begin{flushleft}
		
		There are a number of popular ways to transcribe the trajectory. They are often divided into \emph{collocation} and \emph{shooting} methods.
		
		\bigskip
		
		Shooting methods transcribe $\bo{u}(t)$ via finite number of parameters (spline coefficients or values of $\bo{u}$ at particular time intervals), and then compute $\bo{x}(t)$ via integration.
		
		\bigskip
		
		Collocation methods transcribe both $\bo{u}(t)$ and $\bo{x}(t)$ via finite number of parameters (spline coefficients for $\bo{u}(t)$ and $\bo{x}(t)$), and use dynamics equations as constraints.		
		
	\end{flushleft}
 \end{frame}




\begin{frame}{Direct collocation}
	% \framesubtitle{Parameter estimation}
	\begin{flushleft}
		
		Below is an example of a simple direct collocation, where the $\bo{x}(t)$ and $\bo{u}(t)$ are discretized at time nodes $t_1, \ ..., \ t_n$ as $(\bo{x}_1, \bo{u}_1), \ ..., \ (\bo{x}_n, \bo{u}_n)$.
		%
		\begin{equation*}
			\begin{aligned}
				& \underset{\substack{\bo{x}_1, \ ..., \ \bo{x}_n, \\ \bo{u}_1, \ ..., \ \bo{u}_{n-1}}}{\text{minimize}}
				& & \sum\limits_{i=1}^{n-1} \left( \bo{x}_i\T \bo{Q}_i \bo{x}_i + \bo{u}_i\T \bo{R}_i \bo{u}_i \right) + (\bo{x}_n - \bo{x}_d)\T \bo{Q}_n (\bo{x}_n - \bo{x}_d), \\
				& \text{subject to:}
				& & \frac{\bo{x}_{i+1} - \bo{x}_i}{\Delta t_i} = 
				\bo{f}(\bo{x}_i, \bo{u}_i) \\
				& & & \bo{x}_{min} \leq \bo{x}_i \leq \bo{x}_{max} \\
				& & & \bo{u}_{min} \leq \bo{u}_i \leq \bo{u}_{max}
			\end{aligned}
		\end{equation*}
		
		Approximating derivative as a difference does not strike us as highly accurate in this case. We can do a little better with the following constraint:
		%
		\begin{equation}
		\bo{x}_{i+1} = \frac{\Delta t_i}{2} (\bo{f}(\bo{x}_i, \bo{u}_i) + \bo{f}(\bo{x}_{i+1}, \bo{u}_{i+1}))
		\end{equation} 
		
	\end{flushleft}
\end{frame}




\begin{frame}{Direct collocation - linear discrete systems}
	% \framesubtitle{Parameter estimation}
	\begin{flushleft}
		
		In case of a linear discrete system $\bo{x}_{i+1} = \bo{A}_i \bo{x}_i + \bo{B}_i \bo{u}_i$, direct collocation becomes the obvious choice, equivalent to MPC. With quadratic cost it becomes a finite-time version of LQR:
		%
		\begin{equation*}
			\begin{aligned}
				& \underset{\substack{\bo{x}_1, \ ..., \ \bo{x}_n, \\ \bo{u}_1, \ ..., \ \bo{u}_{n-1}}}{\text{minimize}}
				& & \sum\limits_{i=1}^{n-1} \left( \bo{x}_i\T \bo{Q}_i \bo{x}_i + \bo{u}_i\T \bo{R}_i \bo{u}_i \right) + (\bo{x}_n - \bo{x}_d)\T \bo{Q}_n (\bo{x}_n - \bo{x}_d), \\
				& \text{subject to:}
				& & \bo{x}_{i+1} = \bo{A}_i \bo{x}_i + \bo{B}_i \bo{u}_i \\
				& & & \bo{x}_{min} \leq \bo{x}_i \leq \bo{x}_{max} \\
				& & & \bo{u}_{min} \leq \bo{u}_i \leq \bo{u}_{max}
			\end{aligned}
		\end{equation*}
		
		Note, that if there were no inequality constraints, this problem would be solved analytically.
		
	\end{flushleft}
\end{frame}




\begin{frame}{Single shooting}
	% \framesubtitle{Parameter estimation}
	\begin{flushleft}
		
		Below is an example of a simple single shooting, where the $\bo{u}(t)$ is discretized at time nodes $t_1, \ ..., \ t_n$ as $\bo{u}_1, \ ..., \  \bo{u}_n$.
		
		\begin{equation}
			\begin{aligned}
				& \underset{\substack{\bo{u}_1, \ ..., \ \bo{u}_n}}{\text{minimize}}
				& & \sum\limits_{i=1}^{n-1} ||\bo{u}_i|| + (\bo{x}(t_f) - \bo{x}_d)\T \bo{Q}_n (\bo{x}(t_f) - \bo{x}_d), \\
				& \text{subject to:}
				& & \bo{x} =  \int\limits_{t_0}^{t_f} \bo{f}(\bo{x}, \bo{u}) dt\\
				& & & \bo{u}_{min} \leq \bo{u}_i \leq \bo{u}_{max}
			\end{aligned}
		\end{equation}
		
		Note that here it is possible to use sophisticated integration schemes, but also it is hard to impose state constraints (joint limits, joint velocity limits, obstacle avoidance, etc.).
		
		
	\end{flushleft}
\end{frame}




\begin{frame}{Read more}
	% \framesubtitle{Parameter estimation}
	\begin{flushleft}
		
		
\textcolor{blue}{\href{http://www.matthewpeterkelly.com/research/MatthewKelly_IntroTrajectoryOptimization_SIAM_Review_2017.pdf}{Matthew Kelly Intro Trajectory Optimization.}}


\end{flushleft}
\end{frame}


\begin{frame}{Thank you!}
\centerline{Lecture slides are available via Moodle.}
\bigskip
\centerline{You can help improve these slides at:}
\centerline{\mygit}
\bigskip
\centerline{Check Moodle for additional links, videos, textbook suggestions.}
\bigskip

\centerline{\textcolor{black}{\qrcode[height=1.6in]{https://github.com/SergeiSa/Fundamentals-of-robotics-2022}}}

\end{frame}

\end{document}
