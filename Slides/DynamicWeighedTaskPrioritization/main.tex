\documentclass{beamer}

\pdfmapfile{+sansmathaccent.map}


\mode<presentation>
{
  \usetheme{Warsaw} % or try Darmstadt, Madrid, Warsaw, Rochester, CambridgeUS, ...
  \usecolortheme{seahorse} % or try seahorse, beaver, crane, wolverine, ...
  \usefonttheme{serif}  % or try serif, structurebold, ...
  \setbeamertemplate{navigation symbols}{}
  \setbeamertemplate{caption}[numbered]
} 


%%%%%%%%%%%%%%%%%%%%%%%%%%%%
% itemize settings

\definecolor{mypink}{RGB}{255, 30, 80}
\definecolor{mydarkblue}{RGB}{60, 160, 255}
\definecolor{mydarkred}{RGB}{160, 30, 30}
\definecolor{mylightred}{RGB}{255, 150, 150}
\definecolor{myred}{RGB}{200, 110, 110}
\definecolor{myblackblue}{RGB}{40, 40, 120}
\definecolor{myblackred}{RGB}{120, 40, 40}
\definecolor{myblue}{RGB}{240, 240, 255}
\definecolor{mygreen}{RGB}{0, 200, 0}
\definecolor{mygreen2}{RGB}{205, 255, 200}
\definecolor{mygray}{gray}{0.8}
% \definecolor{mydarkgray}{gray}{0.4}
\definecolor{mydarkgray}{RGB}{80, 80, 160}

\setbeamertemplate{itemize items}[default]

\setbeamertemplate{itemize item}{\color{myblackred}$\blacksquare$}
\setbeamertemplate{itemize subitem}{\color{mydarkred}$\blacktriangleright$}
\setbeamertemplate{itemize subsubitem}{\color{mygray}$\blacksquare$}

\setbeamercolor{palette quaternary}{fg=white,bg=myred} %mydarkgray
\setbeamercolor{titlelike}{parent=palette quaternary}

\setbeamercolor{palette quaternary2}{fg=white,bg=mydarkred}%black myblue
\setbeamercolor{frametitle}{parent=palette quaternary2}

\setbeamerfont{frametitle}{size=\Large,series=\scshape}
\setbeamerfont{framesubtitle}{size=\normalsize,series=\upshape}





%%%%%%%%%%%%%%%%%%%%%%%%%%%%
% block settings

\setbeamercolor{block title}{bg=red!30,fg=black}

\setbeamercolor*{block title example}{bg=mygreen!40!white,fg=black}

\setbeamercolor*{block body example}{fg= black, bg= mygreen2}


%%%%%%%%%%%%%%%%%%%%%%%%%%%%
% URL settings
\hypersetup{
    colorlinks=true,
    linkcolor=blue,
    filecolor=blue,      
    urlcolor=blue,
}

%%%%%%%%%%%%%%%%%%%%%%%%%%

\renewcommand{\familydefault}{\rmdefault}

\usepackage{amsmath}
\usepackage{mathtools}
\usepackage{mathrsfs}


\usepackage{subcaption}

\usepackage{qrcode}

\DeclareMathOperator*{\argmin}{arg\,min}
\newcommand{\bo}[1] {\mathbf{#1}}

\newcommand{\dx}[1] {\dot{\mathbf{#1}}}
\newcommand{\ma}[4] {\begin{bmatrix}
    #1 & #2 \\ #3 & #4
    \end{bmatrix}}
\newcommand{\myvec}[2] {\begin{bmatrix}
    #1 \\ #2
    \end{bmatrix}}
\newcommand{\myvecT}[2] {\begin{bmatrix}
    #1 & #2
    \end{bmatrix}}
 
 \newcommand{\R}{\mathbb{R}} 
 \newcommand{\T}{^\top}     
    

\newcommand{\mydate}{Fall 2022}
\newcommand{\mygit}{\textcolor{blue}{\href{https://github.com/SergeiSa/Fundamentals-of-robotics-2022}{github.com/SergeiSa/Fundamentals-of-robotics-2022}}}


\newcommand{\bref}[2] {\textcolor{blue}{\href{#1}{#2}}}




%%%%%%%%%%%%%%%%%%%%%%%%%%%%
% code settings

\usepackage{listings}
\usepackage{color}
% \definecolor{mygreen}{rgb}{0,0.6,0}
% \definecolor{mygray}{rgb}{0.5,0.5,0.5}
\definecolor{mymauve}{rgb}{0.58,0,0.82}
\lstset{ 
  backgroundcolor=\color{white},   % choose the background color; you must add \usepackage{color} or \usepackage{xcolor}; should come as last argument
  basicstyle=\footnotesize,        % the size of the fonts that are used for the code
  breakatwhitespace=false,         % sets if automatic breaks should only happen at whitespace
  breaklines=true,                 % sets automatic line breaking
  captionpos=b,                    % sets the caption-position to bottom
  commentstyle=\color{mygreen},    % comment style
  deletekeywords={...},            % if you want to delete keywords from the given language
  escapeinside={\%*}{*)},          % if you want to add LaTeX within your code
  extendedchars=true,              % lets you use non-ASCII characters; for 8-bits encodings only, does not work with UTF-8
  firstnumber=0000,                % start line enumeration with line 0000
  frame=single,	                   % adds a frame around the code
  keepspaces=true,                 % keeps spaces in text, useful for keeping indentation of code (possibly needs columns=flexible)
  keywordstyle=\color{blue},       % keyword style
  language=Octave,                 % the language of the code
  morekeywords={*,...},            % if you want to add more keywords to the set
  numbers=left,                    % where to put the line-numbers; possible values are (none, left, right)
  numbersep=5pt,                   % how far the line-numbers are from the code
  numberstyle=\tiny\color{mygray}, % the style that is used for the line-numbers
  rulecolor=\color{black},         % if not set, the frame-color may be changed on line-breaks within not-black text (e.g. comments (green here))
  showspaces=false,                % show spaces everywhere adding particular underscores; it overrides 'showstringspaces'
  showstringspaces=false,          % underline spaces within strings only
  showtabs=false,                  % show tabs within strings adding particular underscores
  stepnumber=2,                    % the step between two line-numbers. If it's 1, each line will be numbered
  stringstyle=\color{mymauve},     % string literal style
  tabsize=2,	                   % sets default tabsize to 2 spaces
  title=\lstname                   % show the filename of files included with \lstinputlisting; also try caption instead of title
}


%%%%%%%%%%%%%%%%%%%%%%%%%%%%
% URL settings
\hypersetup{
    colorlinks=false,
    linkcolor=blue,
    filecolor=blue,      
    urlcolor=blue,
}

%%%%%%%%%%%%%%%%%%%%%%%%%%

%%%%%%%%%%%%%%%%%%%%%%%%%%%%
% tikz settings

\usepackage{tikz}
\tikzset{every picture/.style={line width=0.75pt}}

\newcommand{\iH} {\mathbf{H}^{-1}}
\newcommand{\JH} {\bo{J}^\#_\bo{H}}


\title{Dynamically-consistent pseudoinverse and task prioritization}
\subtitle{Fundamentals of Robotics, Lecture 13}
\author{by Sergei Savin}
\centering
\date{\mydate}



\begin{document}
\maketitle





\begin{frame}{Tasks}
	% \framesubtitle{Parameter estimation}
	\begin{flushleft}
		
		A task is an equation that we wish robot trajectory to abide to. For example, let us consider a task $\bo{r}_t(t, \bo{q})$:
		
		\begin{equation}
			\bo{r}_t(t, \bo{q}) = \bo{r}_e(\bo{q}) - \bo{r}^*_e(t) = 0
		\end{equation} 
	%
	where $\bo{r}_e(\bo{q})$ is the position of the end effector subject to generalized coordinates, and $\bo{r}^*_e(t)$ is the desired position of the end effector.
	
		
	\end{flushleft}
\end{frame}


\begin{frame}{Tasks}
	% \framesubtitle{Parameter estimation}
	\begin{flushleft}
		
		Tasks are very similar to constraints. A task of the form $\bo{r}_t(\bo{q}) = \bo{r}^*_t(t)$ can be differentiated twice:
		
		\begin{align}
			\bo{J} \dot{\bo{q}} = \dot{\bo{r}}^*_t \\
			\bo{J}\ddot{\bo{q}} + \dot{\bo{J}} \dot{\bo{q}} = \ddot{\bo{r}}^*_t
		\end{align}
		
		However, there are no naturally occurring forces that implement the tasks (like reaction forces did for constraints). It is the control implemented by us that acts as a reaction force pushing the system to follow the task.
		
	\end{flushleft}
\end{frame}





\begin{frame}{Tasks and dynamics}
	% \framesubtitle{Parameter estimation}
	\begin{flushleft}
		
		How would the robot behave if a task is obeyed? On the most basic level, it would look like this:
		
		\begin{equation}
			\begin{cases}
				\bo{H} \ddot{\bo{q}} + \bo{C} \dot{\bo{q}} + \bo{g} = \tau \\
				\bo{J}\ddot{\bo{q}} + \dot{\bo{J}} \dot{\bo{q}} = \ddot{\bo{r}}^*_t
			\end{cases}
		\end{equation}		
		
		Assuming that $\tau = \tau^* + \bo{J}\T \lambda$, we can rewrite it as:
		
		\begin{equation}
	\begin{cases}
		\bo{H} \ddot{\bo{q}}  = \tau^* + \bo{J}\T \lambda - (\bo{C} \dot{\bo{q}} + \bo{g}) \\
		\bo{J}\ddot{\bo{q}} = \ddot{\bo{r}}^*_t - \dot{\bo{J}} \dot{\bo{q}}
	\end{cases}
		\end{equation}				
		
		Finally, we define $\tau^* = \bo{C} \dot{\bo{q}} + \bo{g}$ and $\bo{y} = \ddot{\bo{r}}^*_t - \dot{\bo{J}} \dot{\bo{q}}$, giving us:
		
		\begin{equation}
	\begin{cases}
		\bo{H} \ddot{\bo{q}}  =  \bo{J}\T \lambda \\
		\bo{J}\ddot{\bo{q}}    = \bo{y}
	\end{cases}
		\end{equation}		
		
	\end{flushleft}
\end{frame}




\begin{frame}{Solving for a single task}
	% \framesubtitle{Parameter estimation}
	\begin{flushleft}
		
		The equation $\begin{cases}
			\bo{H} \ddot{\bo{q}}  =  \bo{J}\T \lambda \\
			\bo{J}\ddot{\bo{q}}    = \bo{y}
		\end{cases}$ can be solved:
		
		\begin{align}
			\ddot{\bo{q}}  &=  \bo{H}^{-1}\bo{J}\T \lambda  
			\\
			\bo{J}\bo{H}^{-1}\bo{J}\T \lambda    &= \bo{y} 
			\\
			 \lambda    &= (\bo{J}\bo{H}^{-1}\bo{J}\T)^{-1}\bo{y} 
			 \\
			 \ddot{\bo{q}}  &=  \bo{H}^{-1}\bo{J}\T (\bo{J}\bo{H}^{-1}\bo{J}\T)^{-1}\bo{y} 
			 \\
			 \ddot{\bo{q}}  &=  \bo{H}^{-1}\bo{J}\T (\bo{J}\bo{H}^{-1}\bo{J}\T)^{-1}
			 (\ddot{\bo{r}}^*_t - \dot{\bo{J}} \dot{\bo{q}})
		\end{align}		
		
		So, the control law is:
		
		\begin{align}
			\ddot{\bo{q}} &= \bo{J}^\#_\bo{H}  \bo{y} \\
			\bo{J}^\#_\bo{H} &= \bo{H}^{-1}\bo{J}\T (\bo{J}\bo{H}^{-1}\bo{J}\T)^{-1}
		\end{align}
	%
	where $\bo{J}^\#_\bo{H}$ is a \emph{dynamically-consistent pseudoinverse}.
		
	\end{flushleft}
\end{frame}




\begin{frame}{Properties of the dynamic pseudoinverse}
	% \framesubtitle{Parameter estimation}
	\begin{flushleft}
		
		Transpose of the weighted inverse has form:
		
		\begin{align}
			(\JH)\T = (\bo{J} \iH \bo{J}\T)^{-1} \bo{J} \iH
		\end{align}
		
		Consider matrix $\bo{C}$ acting a little like column space projector:
		%
		\begin{align}
			\bo{C} = \bo{J} \JH = \bo{J} \iH \bo{J}\T (\bo{J} \iH \bo{J}\T)^{-1}  = \bo{I}
		\end{align}
		
	\end{flushleft}
\end{frame}



\begin{frame}{Properties of the dynamic pseudoinverse}
	% \framesubtitle{Parameter estimation}
	\begin{flushleft}
		
		Consider matrix $\bo{N}$ acting a little like null space projector:
		%
		\begin{align}
			\bo{N} = \bo{I} - \JH \bo{J} 
			= \bo{I} - \iH \bo{J}\T (\bo{J} \iH \bo{J}\T)^{-1} \bo{J}
		\end{align}
		
		Let us prove that $\bo{N}\JH = 0$.
		%
		\begin{align}
			e &= (\bo{I} - \JH \bo{J}) \JH 
			= \JH - \JH \bo{J}\JH
		\end{align}
		
		Since $\bo{J} \JH = \bo{I}$, we get $e = \JH - \JH = 0$.
		
		\bigskip
		
		Let us prove that $\bo{J} \bo{N} = 0$.
		%
		\begin{align}
			e = \bo{J} \bo{N}
			= \bo{J} (\bo{I} - \JH \bo{J}) =\\
			 = \bo{J} - \bo{J} \JH \bo{J} 
			= \bo{J} - \bo{J} = 0
		\end{align}
		
	\end{flushleft}
\end{frame}




\begin{frame}{Two tasks}
	% \framesubtitle{Parameter estimation}
	\begin{flushleft}
		
		Consider a situation when we have two tasks. Assuming we can accommodate both, it would be great to solve for them sequentially.
		
		We remember that the solution to the following system is:
		
		\begin{equation}
			\begin{cases}
				\bo{H} \bo{x} = \bo{J}\T \lambda \\
				\bo{J} \bo{x} = \bo{y}
			\end{cases}
		\end{equation}
		%
		is $\bo{x} = \JH \bo{y}$.  Now we fix $\bo{x}$ and $\lambda = (\bo{J} \iH \bo{J}\T)^{-1} \bo{y}$ and introduce a new system:
		
		\begin{equation}
			\begin{cases}
				\bo{H} (\bo{x} + \chi) = \bo{J}\T \lambda + (\bo{G}\bo{N})\T \beta\\
				\bo{G}\bo{N} (\bo{x} + \chi) = \bo{z} \\
				\bo{J} (\bo{x} + \chi) = \bo{y}
			\end{cases}
		\end{equation}
		%
		where $\bo{N} = \bo{I} - \textcolor{mydarkblue}{\JH} \bo{J} = \bo{I} - \textcolor{mydarkblue}{\iH \bo{J}\T (\bo{J} \iH \bo{J}\T)^{-1} }\bo{J}$. 		Let us find $\chi$. 
		
	\end{flushleft}
\end{frame}




\begin{frame}{Two tasks}
	% \framesubtitle{Parameter estimation}
	\begin{flushleft}
		
		Let us first study $\bo{G}\bo{N}\bo{x}$:
		%
		\begin{align}
			e = \bo{G}\bo{N}\bo{x} 
			 = \bo{G} \textcolor{red}{\bo{N}\JH} \bo{y}=0
		\end{align}
		
		We found that $\bo{G}\bo{N}\bo{x} = 0$. Hence we can simplify:
		%
		\begin{equation}
			\begin{cases}
				\bo{H} (\bo{x} + \chi) = \bo{J}\T \lambda + (\bo{G}\bo{N})\T \beta\\
				\bo{G}\bo{N} \chi = \bo{z} \\
				\bo{J} (\bo{x} + \chi) = \bo{y}
			\end{cases}
		\end{equation}
		
		Additionally, we observe that $\bo{H} \bo{x} = \bo{J}\T \lambda$ so we simplify:
		%
		\begin{equation}
			\begin{cases}
				\bo{H} \chi = (\bo{G}\bo{N})\T \beta\\
				\bo{G}\bo{N} \chi = \bo{z} \\
				\bo{J} \chi = 0
			\end{cases}
		\end{equation}
		
	\end{flushleft}
\end{frame}



\begin{frame}{Two tasks}
	% \framesubtitle{Parameter estimation}
	\begin{flushleft}
		
		So, we have:
		
		\begin{equation}
			\begin{cases}
				\bo{H} \chi = (\bo{G}\bo{N})\T \beta\\
				\bo{G}\bo{N} \chi = \bo{z} \\
				\bo{J} \chi = 0
			\end{cases}
		\end{equation}
		
		We solve for $\chi$:
		
		\begin{align}
			\chi = \iH (\bo{G}\bo{N})\T \beta \\
			\bo{G}\bo{N} \iH (\bo{G}\bo{N})\T \beta = \bo{z} \\
			\beta = (\bo{G}\bo{N} \iH (\bo{G}\bo{N})\T)^{-1}\bo{z} \\
			\chi = \iH ( \textcolor{mygreen}{\bo{G}\bo{N}} )\T (\textcolor{mygreen}{\bo{G}\bo{N}} \iH (\textcolor{mygreen}{\bo{G}\bo{N}})\T)^{-1}\bo{z} \\
			\chi = (\bo{G}\bo{N})_\bo{H}^\# \bo{z}
		\end{align}
		
	\end{flushleft}
\end{frame}



\begin{frame}{Two tasks}
	% \framesubtitle{Parameter estimation}
	\begin{flushleft}
		
		Thus, we get a rather straight-forward way to solve for multiple tasks:
		
		\begin{itemize}
			\item first task: $\bo{x} = \bo{J}^\#_\bo{H}  \bo{y}$
			\item second task: $\chi = (\bo{G}\bo{N})_\bo{H}^\# \bo{z}$
			\item solution: $\ddot{\bo{q}} = \bo{x} + \chi$
		\end{itemize}
		
		\bigskip
		
		So, adding a new task, we simply multiply the jacobian by the null space "projector" and use the same dynamically consistent pseudoinverse. The resulting generalized acceleration is sum of the two components we found independently.
		
	\end{flushleft}
\end{frame}



\begin{frame}{Prove consistently}
	% \framesubtitle{Parameter estimation}
	\begin{flushleft}
		
		We can prove that the solution for the second task we found does not violate the original solution. I.e.,	does it honor the last constraint $\bo{J} \chi = 0$?
		
		\begin{align}
			e &= \bo{J}(\bo{G}\bo{N})_\bo{H}^\# \bo{z} \\
			e &= \bo{J}\iH (\bo{G}\bo{N})\T (\bo{G}\bo{N} \iH (\bo{G}\bo{N})\T)^{-1} \bo{z}\\
			e &= \textcolor{red}{\bo{J}\iH \bo{N}\T}  \bo{G}\T (\bo{G}\bo{N} \iH \bo{N}\T \bo{G}\T)^{-1} \bo{z} \\
			e &= 0
		\end{align}
		
	\end{flushleft}
\end{frame}




\begin{frame}{Prove consistently}
	% \framesubtitle{Parameter estimation}
	\begin{flushleft}
		
		
		Let us prove that $\bo{J}\iH \bo{N}\T = 0$.
		%
		\begin{align}
			e &= \bo{J} \iH \bo{N}\T \\
			e &= \bo{J} \iH (\bo{I} - \JH \bo{J})\T  \\
			e &= \bo{J} \iH (\bo{I} - \bo{J}\T (\JH)\T )  \\
			e &= \bo{J} \iH - \bo{J} \iH \bo{J}\T (\JH)\T   \\
			e &= \bo{J} \iH - \bo{J} \iH \bo{J}\T (\bo{J} \iH \bo{J}\T)^{-1} \bo{J} \iH 
			\\
			e &= (\bo{I} - \bo{J} \iH \bo{J}\T (\bo{J} \iH \bo{J}\T)^{-1}) \bo{J} \iH 
			\\
			e &= (\bo{I} - \bo{I}) \bo{J} \iH 
			\\
			e &= 0 
		\end{align}
		
	\end{flushleft}
\end{frame}



\begin{frame}{The process}
	% \framesubtitle{Parameter estimation}
	\begin{flushleft}
		
		We can put it this way. Sequential weighted inverse allows us to simplify:
		
		$$
		\begin{cases}
			\bo{H} (\bo{x} + \chi) = \bo{J}\T \lambda + (\bo{G}\bo{N})\T \beta\\
			\bo{G}\bo{N} (\bo{x} + \chi) = \bo{z} \\
			\bo{J} (\bo{x} + \chi) = \bo{y}
		\end{cases}
		\longrightarrow
		\begin{cases}
			\bo{H} \chi = (\bo{G}\bo{N})\T \beta\\
			\bo{G}\bo{N} \chi = \bo{z}
		\end{cases}
		$$
		
		And this was possible because we used Jacobian matrix multiplied by generalized null space projector on the right: $\bo{G}\bo{N}$.
		
		
	\end{flushleft}
\end{frame}




\begin{frame}{Read more}
	% \framesubtitle{Parameter estimation}
	\begin{flushleft}
		
		
		\begin{itemize}
			\item Kim, D., Di Carlo, J., Katz, B., Bledt, G. and Kim, S., 2019. Highly dynamic quadruped locomotion via whole-body impulse control and model predictive control. arXiv preprint arXiv:1909.06586.
		\end{itemize}
		
		
	\end{flushleft}
\end{frame}



\begin{frame}{Thank you!}
\centerline{Lecture slides are available via Moodle.}
\bigskip
\centerline{You can help improve these slides at:}
\centerline{\mygit}
\bigskip
\centerline{Check Moodle for additional links, videos, textbook suggestions.}
\bigskip

\centerline{\textcolor{black}{\qrcode[height=1.6in]{https://github.com/SergeiSa/Fundamentals-of-robotics-2022}}}

\end{frame}



\end{document}
