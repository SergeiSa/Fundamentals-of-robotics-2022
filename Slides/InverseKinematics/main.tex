\documentclass{beamer}

\pdfmapfile{+sansmathaccent.map}


\mode<presentation>
{
  \usetheme{Warsaw} % or try Darmstadt, Madrid, Warsaw, Rochester, CambridgeUS, ...
  \usecolortheme{seahorse} % or try seahorse, beaver, crane, wolverine, ...
  \usefonttheme{serif}  % or try serif, structurebold, ...
  \setbeamertemplate{navigation symbols}{}
  \setbeamertemplate{caption}[numbered]
} 


%%%%%%%%%%%%%%%%%%%%%%%%%%%%
% itemize settings

\definecolor{mypink}{RGB}{255, 30, 80}
\definecolor{mydarkblue}{RGB}{60, 160, 255}
\definecolor{mydarkred}{RGB}{160, 30, 30}
\definecolor{mylightred}{RGB}{255, 150, 150}
\definecolor{myred}{RGB}{200, 110, 110}
\definecolor{myblackblue}{RGB}{40, 40, 120}
\definecolor{myblackred}{RGB}{120, 40, 40}
\definecolor{myblue}{RGB}{240, 240, 255}
\definecolor{mygreen}{RGB}{0, 200, 0}
\definecolor{mygreen2}{RGB}{205, 255, 200}
\definecolor{mygray}{gray}{0.8}
% \definecolor{mydarkgray}{gray}{0.4}
\definecolor{mydarkgray}{RGB}{80, 80, 160}

\setbeamertemplate{itemize items}[default]

\setbeamertemplate{itemize item}{\color{myblackred}$\blacksquare$}
\setbeamertemplate{itemize subitem}{\color{mydarkred}$\blacktriangleright$}
\setbeamertemplate{itemize subsubitem}{\color{mygray}$\blacksquare$}

\setbeamercolor{palette quaternary}{fg=white,bg=myred} %mydarkgray
\setbeamercolor{titlelike}{parent=palette quaternary}

\setbeamercolor{palette quaternary2}{fg=white,bg=mydarkred}%black myblue
\setbeamercolor{frametitle}{parent=palette quaternary2}

\setbeamerfont{frametitle}{size=\Large,series=\scshape}
\setbeamerfont{framesubtitle}{size=\normalsize,series=\upshape}





%%%%%%%%%%%%%%%%%%%%%%%%%%%%
% block settings

\setbeamercolor{block title}{bg=red!30,fg=black}

\setbeamercolor*{block title example}{bg=mygreen!40!white,fg=black}

\setbeamercolor*{block body example}{fg= black, bg= mygreen2}


%%%%%%%%%%%%%%%%%%%%%%%%%%%%
% URL settings
\hypersetup{
    colorlinks=true,
    linkcolor=blue,
    filecolor=blue,      
    urlcolor=blue,
}

%%%%%%%%%%%%%%%%%%%%%%%%%%

\renewcommand{\familydefault}{\rmdefault}

\usepackage{amsmath}
\usepackage{mathtools}
\usepackage{mathrsfs}


\usepackage{subcaption}

\usepackage{qrcode}

\DeclareMathOperator*{\argmin}{arg\,min}
\newcommand{\bo}[1] {\mathbf{#1}}

\newcommand{\dx}[1] {\dot{\mathbf{#1}}}
\newcommand{\ma}[4] {\begin{bmatrix}
    #1 & #2 \\ #3 & #4
    \end{bmatrix}}
\newcommand{\myvec}[2] {\begin{bmatrix}
    #1 \\ #2
    \end{bmatrix}}
\newcommand{\myvecT}[2] {\begin{bmatrix}
    #1 & #2
    \end{bmatrix}}
 
 \newcommand{\R}{\mathbb{R}} 
 \newcommand{\T}{^\top}     
    

\newcommand{\mydate}{Fall 2022}
\newcommand{\mygit}{\textcolor{blue}{\href{https://github.com/SergeiSa/Fundamentals-of-robotics-2022}{github.com/SergeiSa/Fundamentals-of-robotics-2022}}}


\newcommand{\bref}[2] {\textcolor{blue}{\href{#1}{#2}}}




%%%%%%%%%%%%%%%%%%%%%%%%%%%%
% code settings

\usepackage{listings}
\usepackage{color}
% \definecolor{mygreen}{rgb}{0,0.6,0}
% \definecolor{mygray}{rgb}{0.5,0.5,0.5}
\definecolor{mymauve}{rgb}{0.58,0,0.82}
\lstset{ 
  backgroundcolor=\color{white},   % choose the background color; you must add \usepackage{color} or \usepackage{xcolor}; should come as last argument
  basicstyle=\footnotesize,        % the size of the fonts that are used for the code
  breakatwhitespace=false,         % sets if automatic breaks should only happen at whitespace
  breaklines=true,                 % sets automatic line breaking
  captionpos=b,                    % sets the caption-position to bottom
  commentstyle=\color{mygreen},    % comment style
  deletekeywords={...},            % if you want to delete keywords from the given language
  escapeinside={\%*}{*)},          % if you want to add LaTeX within your code
  extendedchars=true,              % lets you use non-ASCII characters; for 8-bits encodings only, does not work with UTF-8
  firstnumber=0000,                % start line enumeration with line 0000
  frame=single,	                   % adds a frame around the code
  keepspaces=true,                 % keeps spaces in text, useful for keeping indentation of code (possibly needs columns=flexible)
  keywordstyle=\color{blue},       % keyword style
  language=Octave,                 % the language of the code
  morekeywords={*,...},            % if you want to add more keywords to the set
  numbers=left,                    % where to put the line-numbers; possible values are (none, left, right)
  numbersep=5pt,                   % how far the line-numbers are from the code
  numberstyle=\tiny\color{mygray}, % the style that is used for the line-numbers
  rulecolor=\color{black},         % if not set, the frame-color may be changed on line-breaks within not-black text (e.g. comments (green here))
  showspaces=false,                % show spaces everywhere adding particular underscores; it overrides 'showstringspaces'
  showstringspaces=false,          % underline spaces within strings only
  showtabs=false,                  % show tabs within strings adding particular underscores
  stepnumber=2,                    % the step between two line-numbers. If it's 1, each line will be numbered
  stringstyle=\color{mymauve},     % string literal style
  tabsize=2,	                   % sets default tabsize to 2 spaces
  title=\lstname                   % show the filename of files included with \lstinputlisting; also try caption instead of title
}


%%%%%%%%%%%%%%%%%%%%%%%%%%%%
% URL settings
\hypersetup{
    colorlinks=false,
    linkcolor=blue,
    filecolor=blue,      
    urlcolor=blue,
}

%%%%%%%%%%%%%%%%%%%%%%%%%%

%%%%%%%%%%%%%%%%%%%%%%%%%%%%
% tikz settings

\usepackage{tikz}
\tikzset{every picture/.style={line width=0.75pt}}


\title{Inverse Kinematics}
\subtitle{Fundamentals of Robotics, Lecture 6}
\author{by Sergei Savin}
\centering
\date{\mydate}



\begin{document}
\maketitle





\begin{frame}{Velocity problem}
	\begin{flushleft}
		
		Given a point (e.g. end effector) $K$ with position given by vector $\bo{r}_K(\bo{q})$, we can find its velocity:
		
		\begin{equation}
			\dot{\bo{r}}_K(\bo{q}) = \frac{\partial \bo{r}_K}{\partial  \bo{q}} \dot{\bo{q}}
		\end{equation}
		
		Let us introduce notation:
		
		\begin{align}
			\bo{J}_K = \frac{\partial \bo{r}_K}{\partial  \bo{q}} \\
			\bo{v}_K = \dot{\bo{r}}_K(\bo{q})
		\end{align}
	
		Thus we get:
		
		\begin{equation}
			\bo{v}_K = \bo{J}_K \dot{\bo{q}}
		\end{equation}		
		
	\end{flushleft}
\end{frame}




\begin{frame}{Velocity problem}
	\begin{flushleft}
		
		Given $\bo{v}_K = \bo{J}_K \dot{\bo{q}}$, can we find least-residual solution to this problem? Yes!
		
		\begin{equation}
			\dot{\bo{q}}^* = \bo{J}_K^+ \bo{v}_K
		\end{equation}		
	
		Is this the only solution? No. All solutions are:
	
		\begin{equation}
			\dot{\bo{q}}^* = \bo{J}_K^+ \bo{v}_K + \bo{N} \bo{z}
		\end{equation}			
	%
	where $ \bo{N} = \text{null}(\bo{J}_K)$ and $\bo{z}$ are null space coordinates.
	
	Alternatively, we can use a projector to do the same thing with less new notation:
	
		\begin{equation}
			\dot{\bo{q}}^*= \bo{J}_K^+ \bo{v}_K + (\bo{I} - \bo{J}_K^+ \bo{J}_K)  \dot{\bo{q}}
		\end{equation}		
	
	where $\bo{I} - \bo{J}_K^+ \bo{J}_K$ is a null space projector.
		
	\end{flushleft}
\end{frame}




\begin{frame}{Optimal velocity problem}
	\begin{flushleft}
		
		Now let us find closest joint velocity to $\dot{\bo{q}}_0$ that solves the velocity problem $\bo{v}_K = \bo{J}_K \dot{\bo{q}}$:
		
\begin{equation}
	\begin{aligned}
		& \underset{\dot{\bo{q}}}{\text{minimize}}
		& & || \dot{\bo{q}} - \dot{\bo{q}}_0 ||, \\
		& \text{subject to}
		& & \bo{v}_K = \bo{J}_K \dot{\bo{q}}
	\end{aligned}
\end{equation}
		
		We can solve it by first finding all solutions:
		
		\begin{equation}
			\dot{\bo{q}} = \bo{J}_K^+ \bo{v}_K + \bo{N} \bo{z}
		\end{equation}			
	
	Then we minimize cost function in terms of the null space variable $\bo{z}$:
	
		\begin{equation}
			|| \bo{J}_K^+ \bo{v}_K + \bo{N} \bo{z} - \dot{\bo{q}}_0 || \rightarrow min
		\end{equation}	
		
	\end{flushleft}
\end{frame}




\begin{frame}{Optimal velocity problem}
	\begin{flushleft}
		
		First we simplify $|| \bo{J}_K^+ \bo{v}_K + \bo{N} \bo{z} - \dot{\bo{q}}_0 || \rightarrow min$ with notation
		
		\begin{equation}
			\bo{c} = \bo{J}_K^+ \bo{v}_K -  \dot{\bo{q}}_0
		\end{equation}			
		
		Then we square $|| \bo{N} \bo{z} + \bo{c} ||$, and consider its derivative:
		%
		\begin{align}
			(\bo{N} \bo{z} + \bo{c})\T (\bo{N} \bo{z} + \bo{c}) \rightarrow min \\
			\frac{\partial }{\partial \bo{z}} (\bo{N} \bo{z} + \bo{c})\T (\bo{N} \bo{z} + \bo{c}) = 0 \\
			2 \bo{z}\T \bo{N}\T \bo{N} + 2 \bo{c}\T \bo{N} = 0 \\
			\bo{z} = -( \bo{N}\T \bo{N})^{-1} \bo{N}\T \bo{c}
		\end{align}	
	
		Knowing that $\dot{\bo{q}} = \bo{J}_K^+ \bo{v}_K + \bo{N} \bo{z}$ we get:
	
		\begin{align}
			\dot{\bo{q}} = \bo{J}_K^+ \bo{v}_K - \bo{N}  ( \bo{N}\T \bo{N})^{-1} \bo{N}\T \bo{c}
		\end{align}		
		
	\end{flushleft}
\end{frame}




\begin{frame}{Optimal velocity problem}
	\begin{flushleft}
		
		Let us examine  the solution we obtained:
		%
		\begin{align}
			\dot{\bo{q}} =  \bo{J}_K^+ \bo{v}_K - \bo{N}  ( \bo{N}\T \bo{N})^{-1} \bo{N}\T (\bo{J}_K^+ \bo{v}_K -  \dot{\bo{q}}_0) 
			\\
			\dot{\bo{q}} = (\bo{I} - \bo{N}  ( \bo{N}\T \bo{N})^{-1} \bo{N}\T)\bo{J}_K^+ \bo{v}_K +  
			\bo{N}  ( \bo{N}\T \bo{N})^{-1} \bo{N}\T\dot{\bo{q}}_0
		\end{align}				
	
		Let us examine the matrices:
		%
		\begin{align}
			\bo{P}_N = \bo{N}  ( \bo{N}\T \bo{N})^{-1} \bo{N}\T
		\\
			\bo{P}_R = \bo{I} - \bo{N}  ( \bo{N}\T \bo{N})^{-1} \bo{N}\T =  \bo{I} - \bo{P}_N \\
			\dot{\bo{q}} =\bo{P}_R \bo{J}_K^+ \bo{v}_K +  
			\bo{P}_N \dot{\bo{q}}_0
		\end{align}	

		where $\bo{P}_N$ is column space projector for $\bo{N}$, hence it is a null space projector for the jacobian $\bo{J}_K$. And $ \bo{I} - \bo{P}_N$ is a projector to the orthogonal compliment, hence it is row space projector.
		
	\end{flushleft}
\end{frame}



\begin{frame}{Optimal velocity problem}
	\begin{flushleft}
		
		So, we have eq. $\dot{\bo{q}} =\bo{P}_R \bo{J}_K^+ \bo{v}_K +  
		\bo{P}_N \dot{\bo{q}}_0$ and we have null space projector $\bo{P}_N$ and row space projector $\bo{P}_R$.
		
		\bigskip
		
		We know that pseudoinverse lies in the column space, so:
		
		\begin{equation}
			 \bo{P}_R \bo{J}_K^+ = \bo{J}_K^+
		\end{equation}
	
		Also we know that null space projector can be found as $\bo{P}_N = \bo{I} - \bo{J}_K^+\bo{J}_K$:
		
		\begin{equation}
			\dot{\bo{q}} =\bo{J}_K^+ \bo{v}_K +  
			(\bo{I} - \bo{J}_K^+\bo{J}_K) \dot{\bo{q}}_0
		\end{equation}		
	
		Notice, this is almost exactly the same as what we found before. We can interpret it as "the solution is given by row-space least squares solution, plus null space projection of $\dot{\bo{q}}_0$".
		
	\end{flushleft}
\end{frame}




\begin{frame}{Acceleration problem}
	\begin{flushleft}
		
		Consider second derivative of the position of the point $K$:
		
\begin{equation}
	\ddot{\bo{r}}_K(\bo{q}) = \frac{\partial \bo{r}_K}{\partial  \bo{q}} \ddot{\bo{q}} 
	+ 
	\frac{d}{dt} \left( \frac{\partial \bo{r}_K}{\partial  \bo{q}} \right) \dot{\bo{q}}
\end{equation}		
		
		Defining $\bo{a}_K = \ddot{\bo{r}}_K$ we get:
		
\begin{equation}
	\bo{a}_K = \bo{J}_K \ddot{\bo{q}} 
	+ 
	\dot{\bo{J}}_K \dot{\bo{q}}
\end{equation}	

		The least residual solution is easily found:
		
\begin{equation}
	\ddot{\bo{q}}  = \bo{J}_K^+
	(\bo{a}_K - \dot{\bo{J}}_K \dot{\bo{q}})
\end{equation}						
		
	\end{flushleft}
\end{frame}




\begin{frame}{Acceleration problem}
	\begin{flushleft}
		
		Given $\bo{a}_K = \bo{J}_K \ddot{\bo{q}} + \dot{\bo{J}}_K \dot{\bo{q}}$ let us find acceleration closest to $\ddot{\bo{q}}_0$ that solves the acceleration problem:
		
		\begin{equation}
			\ddot{\bo{q}}  = \bo{J}_K^+ (\bo{a}_K - \dot{\bo{J}}_K \dot{\bo{q}})
			+
			\bo{P}_N \ddot{\bo{q}}_0
		\end{equation}		
	%
	where $\bo{P}_N = \bo{I} -  \bo{J}_K^+ \bo{J}_K$.
	
		
	\end{flushleft}
\end{frame}



\begin{frame}{Position problem}
	\begin{flushleft}
		
		What if we want to find such $\bo{q}^*$ that $\bo{r}_K(\bo{q}^*) = \bo{r}^*$. Can we do it?
		
		\bigskip
		
		Unlike previous, this is not a linear problem. It often involves trigonometric functions, and other nonlinear ones.
		
		\bigskip
		
		Our approach will be to linearize the expression $\bo{r}_K(\bo{q})$ and find its solution via an iterative procedure.
		
	\end{flushleft}
\end{frame}



\begin{frame}{Position problem - update method}
	\begin{flushleft}
		
		Given exact solution $\bo{q}_0$ for problem $\bo{r}_K(\bo{q}_0) = \bo{r}_K(0)$. Then, knowing velocity $\dot{\bo{q}}_0$, then we can find an approximation of the position in the next moment of time:
		
		\begin{equation}
			\frac{\bo{q}_1 - \bo{q}_0}{\Delta t} \approx \dot{\bo{q}}_0
		\end{equation}
	
		\begin{equation}
		\bo{q}_1 \approx \bo{q}_0 + \dot{\bo{q}}_0 \Delta t
		\end{equation}
	
		This works tolerably well, for improvements we can look to other schemes of solving ODEs.
		
		\bigskip
	
		But what if we don't have an exact solution $\bo{q}_0$? After all, it was that which allowed us to use local linearization.
		
	\end{flushleft}
\end{frame}



\begin{frame}{Position problem - general}
	\begin{flushleft}
		
		Given initial guess $\bo{q}_0$ we will try to solve the problem $\bo{r}_K(\bo{q}) = \bo{r}_K^*$. First let us define discrepancy:
		
		\begin{equation}
			\bo{e}(\bo{q}) = \bo{r}_K(\bo{q}) - \bo{r}_K^*
		\end{equation}		
	
		We define cost function $f =\bo{e}\T \bo{e}$, initial position $\bo{r}_{K, 0} = \bo{r}_K(\bo{q}_0)$, initial discrepancy $\bo{e}_0 =  \bo{r}_{K, 0} - \bo{r}_K^*$ and gen. coordinates displacement $\delta = \bo{q} - \bo{q}_0$ and produce Taylor expansion of the cost:
		
		\begin{equation}
			f \approx \bo{e}_0\T \bo{e}_0 
			+
			\bo{e}_0\T \bo{J}_K \delta
			+
			 \delta \T \bo{J}_K\T \bo{e}_0
			  +
			  \delta \T \bo{J}_K\T \bo{J}_K \delta
		\end{equation}	
	
	Now we take  derivative and set it to zero:
	
			\begin{equation}
		2 \bo{e}_0\T \bo{J}_K
		+
		2 \delta \T \bo{J}_K\T \bo{J}_K = 0
	\end{equation}	
		
	\end{flushleft}
\end{frame}



\begin{frame}{Position problem - general}
	\begin{flushleft}
		
		We obtained expression:
		
		\begin{equation}
			\delta = (\bo{J}_K\T \bo{J}_K)^{-1} \bo{J}_K\T \bo{e}_0
		\end{equation}	
	
	And remembering the substitutions we made we get:
	
	\begin{align}
		\bo{q} - \bo{q}_0 = (\bo{J}_K\T \bo{J}_K)^{-1} \bo{J}_K\T (\bo{r}_{K, 0} - \bo{r}_K^*) \\
		\bo{q} = \bo{q}_0 + \bo{J}_K^+(\bo{r}_{K, 0} - \bo{r}_K^*)
	\end{align}	

	We can use the final expression to update our initial guess, then the-linearize the problem at the new position and repeat the process, until we converge.
		
	\end{flushleft}
\end{frame}



\begin{frame}{Subspaces}
	\begin{flushleft}
		
		How can we check if a velocity $\bo{v}_K$ can be achieved, given $\bo{v}_K = \bo{J}_K \dot{\bo{q}}$?
		
		\bigskip
		
		If $\bo{v}_K$ lies in the column space of $\bo{J}_K$, it is achievable:
		
		\begin{equation}
			(\bo{I} - \bo{J}_K\bo{J}_K^+) \bo{v}_K = 0
		\end{equation}	
	
	
		
	\end{flushleft}
\end{frame}



\begin{frame}{Subspaces}
	\begin{flushleft}
		
		How can we check if a solution $\dot{\bo{q}}$ is minimal-norm, given $\bo{v}_K = \bo{J}_K \dot{\bo{q}}$?
		
		\bigskip
		
		If $\dot{\bo{q}}$ lies in the row space of $\bo{J}_K$, it is minimal:
		
		\begin{equation}
			(\bo{I} - \bo{J}_K^+\bo{J}_K) \dot{\bo{q}} = 0
		\end{equation}	
		
		
		
	\end{flushleft}
\end{frame}




\begin{frame}{Thank you!}
\centerline{Lecture slides are available via Moodle.}
\bigskip
\centerline{You can help improve these slides at:}
\centerline{\mygit}
\bigskip
\centerline{Check Moodle for additional links, videos, textbook suggestions.}
\bigskip

\centerline{\textcolor{black}{\qrcode[height=1.6in]{https://github.com/SergeiSa/Fundamentals-of-robotics-2022}}}

\end{frame}

\end{document}
