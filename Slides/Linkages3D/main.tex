\documentclass{beamer}

\pdfmapfile{+sansmathaccent.map}


\mode<presentation>
{
  \usetheme{Warsaw} % or try Darmstadt, Madrid, Warsaw, Rochester, CambridgeUS, ...
  \usecolortheme{seahorse} % or try seahorse, beaver, crane, wolverine, ...
  \usefonttheme{serif}  % or try serif, structurebold, ...
  \setbeamertemplate{navigation symbols}{}
  \setbeamertemplate{caption}[numbered]
} 


%%%%%%%%%%%%%%%%%%%%%%%%%%%%
% itemize settings

\definecolor{mypink}{RGB}{255, 30, 80}
\definecolor{mydarkblue}{RGB}{60, 160, 255}
\definecolor{mydarkred}{RGB}{160, 30, 30}
\definecolor{mylightred}{RGB}{255, 150, 150}
\definecolor{myred}{RGB}{200, 110, 110}
\definecolor{myblackblue}{RGB}{40, 40, 120}
\definecolor{myblackred}{RGB}{120, 40, 40}
\definecolor{myblue}{RGB}{240, 240, 255}
\definecolor{mygreen}{RGB}{0, 200, 0}
\definecolor{mygreen2}{RGB}{205, 255, 200}
\definecolor{mygray}{gray}{0.8}
% \definecolor{mydarkgray}{gray}{0.4}
\definecolor{mydarkgray}{RGB}{80, 80, 160}

\setbeamertemplate{itemize items}[default]

\setbeamertemplate{itemize item}{\color{myblackred}$\blacksquare$}
\setbeamertemplate{itemize subitem}{\color{mydarkred}$\blacktriangleright$}
\setbeamertemplate{itemize subsubitem}{\color{mygray}$\blacksquare$}

\setbeamercolor{palette quaternary}{fg=white,bg=myred} %mydarkgray
\setbeamercolor{titlelike}{parent=palette quaternary}

\setbeamercolor{palette quaternary2}{fg=white,bg=mydarkred}%black myblue
\setbeamercolor{frametitle}{parent=palette quaternary2}

\setbeamerfont{frametitle}{size=\Large,series=\scshape}
\setbeamerfont{framesubtitle}{size=\normalsize,series=\upshape}





%%%%%%%%%%%%%%%%%%%%%%%%%%%%
% block settings

\setbeamercolor{block title}{bg=red!30,fg=black}

\setbeamercolor*{block title example}{bg=mygreen!40!white,fg=black}

\setbeamercolor*{block body example}{fg= black, bg= mygreen2}


%%%%%%%%%%%%%%%%%%%%%%%%%%%%
% URL settings
\hypersetup{
    colorlinks=true,
    linkcolor=blue,
    filecolor=blue,      
    urlcolor=blue,
}

%%%%%%%%%%%%%%%%%%%%%%%%%%

\renewcommand{\familydefault}{\rmdefault}

\usepackage{amsmath}
\usepackage{mathtools}
\usepackage{mathrsfs}


\usepackage{subcaption}

\usepackage{qrcode}

\DeclareMathOperator*{\argmin}{arg\,min}
\newcommand{\bo}[1] {\mathbf{#1}}

\newcommand{\dx}[1] {\dot{\mathbf{#1}}}
\newcommand{\ma}[4] {\begin{bmatrix}
    #1 & #2 \\ #3 & #4
    \end{bmatrix}}
\newcommand{\myvec}[2] {\begin{bmatrix}
    #1 \\ #2
    \end{bmatrix}}
\newcommand{\myvecT}[2] {\begin{bmatrix}
    #1 & #2
    \end{bmatrix}}
 
 \newcommand{\R}{\mathbb{R}} 
 \newcommand{\T}{^\top}     
    

\newcommand{\mydate}{Fall 2022}
\newcommand{\mygit}{\textcolor{blue}{\href{https://github.com/SergeiSa/Fundamentals-of-robotics-2022}{github.com/SergeiSa/Fundamentals-of-robotics-2022}}}


\newcommand{\bref}[2] {\textcolor{blue}{\href{#1}{#2}}}




%%%%%%%%%%%%%%%%%%%%%%%%%%%%
% code settings

\usepackage{listings}
\usepackage{color}
% \definecolor{mygreen}{rgb}{0,0.6,0}
% \definecolor{mygray}{rgb}{0.5,0.5,0.5}
\definecolor{mymauve}{rgb}{0.58,0,0.82}
\lstset{ 
  backgroundcolor=\color{white},   % choose the background color; you must add \usepackage{color} or \usepackage{xcolor}; should come as last argument
  basicstyle=\footnotesize,        % the size of the fonts that are used for the code
  breakatwhitespace=false,         % sets if automatic breaks should only happen at whitespace
  breaklines=true,                 % sets automatic line breaking
  captionpos=b,                    % sets the caption-position to bottom
  commentstyle=\color{mygreen},    % comment style
  deletekeywords={...},            % if you want to delete keywords from the given language
  escapeinside={\%*}{*)},          % if you want to add LaTeX within your code
  extendedchars=true,              % lets you use non-ASCII characters; for 8-bits encodings only, does not work with UTF-8
  firstnumber=0000,                % start line enumeration with line 0000
  frame=single,	                   % adds a frame around the code
  keepspaces=true,                 % keeps spaces in text, useful for keeping indentation of code (possibly needs columns=flexible)
  keywordstyle=\color{blue},       % keyword style
  language=Octave,                 % the language of the code
  morekeywords={*,...},            % if you want to add more keywords to the set
  numbers=left,                    % where to put the line-numbers; possible values are (none, left, right)
  numbersep=5pt,                   % how far the line-numbers are from the code
  numberstyle=\tiny\color{mygray}, % the style that is used for the line-numbers
  rulecolor=\color{black},         % if not set, the frame-color may be changed on line-breaks within not-black text (e.g. comments (green here))
  showspaces=false,                % show spaces everywhere adding particular underscores; it overrides 'showstringspaces'
  showstringspaces=false,          % underline spaces within strings only
  showtabs=false,                  % show tabs within strings adding particular underscores
  stepnumber=2,                    % the step between two line-numbers. If it's 1, each line will be numbered
  stringstyle=\color{mymauve},     % string literal style
  tabsize=2,	                   % sets default tabsize to 2 spaces
  title=\lstname                   % show the filename of files included with \lstinputlisting; also try caption instead of title
}


%%%%%%%%%%%%%%%%%%%%%%%%%%%%
% URL settings
\hypersetup{
    colorlinks=false,
    linkcolor=blue,
    filecolor=blue,      
    urlcolor=blue,
}

%%%%%%%%%%%%%%%%%%%%%%%%%%

%%%%%%%%%%%%%%%%%%%%%%%%%%%%
% tikz settings

\usepackage{tikz}
\tikzset{every picture/.style={line width=0.75pt}}


\title{3D linkages}
\subtitle{Fundamentals of Robotics, Lecture 3}
\author{by Sergei Savin}
\centering
\date{\mydate}



\begin{document}
\maketitle




\begin{frame}{Frame - introduction}
	%\framesubtitle{1st order}
	\begin{flushleft}
		
		Consider a vector $\bo{v}$. As we noted in the previous lecture, we can express it in terms of its coordinates in a basis of of choice. Or - we can just leave it alone. Examples of vectors in Robotics are linear and angular velocities, as well as accelerations.
		
		\bigskip
		
		Consider a point $P$ in Euclidean space $\mathcal E$. Given some other point $O$ we can associate a vector $\bo{r}_{OP}$ with any point $P \in \mathcal E$ as difference in position of points $P$ and $O$. We call point $O$ origin. Origin and basis together form a \emph{frame}.
		
	\end{flushleft}
\end{frame}

\begin{frame}{Frame - introduction}
	%\framesubtitle{1st order}
	\begin{flushleft}
		
		Notice that to find coordinates of a vector we only need a basis, while for a point we also need an origin - in other words, we need a frame to express a point.
		
	\end{flushleft}
\end{frame}


\begin{frame}{Frame - basic example}
	%\framesubtitle{1st order}
	\begin{flushleft}
		
		Consider the following example. We know coordinates of $\mathbf r_{OP} $ expressed in basis $\mathcal V$, and we know $\mathbf r_{OA}$, also expressed in $\mathcal V$. Can express  $\mathbf r_{AP}$ in $\mathcal V$?
		
		\bigskip
		
		We know that $\mathbf r_{OP} = P - O$ and $\mathbf r_{OA} = A - O$ and $\mathbf r_{AP} = P - A$, hence $\mathbf r_{AP} = \mathbf r_{OP} - \mathbf r_{OA}$, which is true in any basis (of course, to sum coordinates in a meaningful way they should be associated with the same basis).
		
		\bigskip
		
		To denote that $\mathbf r_{AP}$ is expressed in $\mathcal V$ we write it as $^{\mathcal V} \mathbf r_{AP}$.
		
	\end{flushleft}
\end{frame}


\begin{frame}{Sequence of frames}
	%\framesubtitle{1st order}
	\begin{flushleft}
		
		Another example. We have a point $K$ and frames $\mathcal F_1$ and $\mathcal F_2$, as well as a \emph{world frame}. Frame $\mathcal F_1$ given by its origin $O_1$ and basis $\mathcal T_1$. Frame $\mathcal F_2$ given by its origin $O_2$ and basis $\mathcal T_2$. 
		
		\bigskip
		
		We know position of point $O_1$ in the world frame (its coordinates with respect to the origin of the world frame, expressed in the world frame basis) - $^{\mathcal W}\bo{r}_{O1}$. Basis $\mathcal T_1$ is given by a matrix $^{\mathcal W}\mathbf T_1$ whose columns are coordinates of the vectors forming $\mathcal T_1$ expressed in the world frame.
		
		\bigskip
		
		We know position of point $O_2$ in the $\mathcal F_1$ frame - $^{\mathcal T_1}\bo{r}_{O1O2}$. Basis $\mathcal T_2$ is given by a matrix $^{\mathcal T_1}\mathbf T_2$ whose columns are coordinates of the vectors forming $\mathcal T_2$ expressed in the world frame.
		
		\bigskip
		
		Task - express $K$ in world frame given $^{\mathcal T_2}\bo{r}_{O2K}$ - position of $K$ in the frame $\mathcal F_2$. Scary.
		
	\end{flushleft}
\end{frame}


\begin{frame}{Sequence of frames}
	%\framesubtitle{1st order}
	\begin{flushleft}
		
		A good way to deal with these problems is to express every basis in terms of the world frame basis, and then do the same with all points.
		
		\bigskip
		
		First, we already have basis $\mathcal T_1$ in the world frame basis - $^{\mathcal W}\mathbf T_1$ . Now to express $\mathcal T_2$.
		
		\bigskip
		
		Given coordinates $^{\mathcal T_1}\mathbf T_2$ in $\mathcal T_1$ we can find $\mathbf T_2 = \mathbf T_1 \ ^{\mathcal T_1}\mathbf T_2$, hence 
		$^{\mathcal W}\mathbf T_2 = ^{\mathcal W}\mathbf T_1 \ ^{\mathcal T_1}\mathbf T_2$.
		
	\end{flushleft}
\end{frame}



\begin{frame}{Sequence of frames}
	%\framesubtitle{1st order}
	\begin{flushleft}
		
		We already know $^{\mathcal W}\bo{r}_{O1}$ and we know $^{\mathcal T_1}\bo{r}_{O1O2}$. Vector $\bo{r}_{O1O2}$ is therefore:
		
		\begin{align}
			\bo{r}_{O1O2} = \mathbf T_1  \  ^{\mathcal T_1}\bo{r}_{O1O2} \\
			^{\mathcal W}\bo{r}_{O1O2} = ^{\mathcal W}\mathbf T_1  \  ^{\mathcal T_1}\bo{r}_{O1O2} 
		\end{align}
		
		We remember that $\bo{r}_{O1O2} = O_2 - O_1$ and $\bo{r}_{O1} = O_1 - O$ (where $O$ is the world frame origin), and we want to find $\bo{r}_{O2} = O_2 - O$, so:
		
		\begin{align}
			\bo{r}_{O2} = \bo{r}_{O1} + \bo{r}_{O1O2}
			 \\
			^{\mathcal W}\bo{r}_{O2} = ^{\mathcal W}\bo{r}_{O1} + ^{\mathcal W}\bo{r}_{O1O2}
			\\
			^{\mathcal W}\bo{r}_{O2} = ^{\mathcal W}\bo{r}_{O1} + ^{\mathcal W}\mathbf T_1  \  ^{\mathcal T_1}\bo{r}_{O1O2} 
		\end{align}
		
	\end{flushleft}
\end{frame}



\begin{frame}{Sequence of frames}
	%\framesubtitle{1st order}
	\begin{flushleft}
		
		We know $^{\mathcal T_2}\bo{r}_{O2K}$, so vector $\bo{r}_{O2K}$ can be found as:
		
		\begin{align}
			\bo{r}_{O2K} = \mathbf T_2 \  ^{\mathcal T_2}\bo{r}_{O2K}
			 \\
			 ^{\mathcal W} \bo{r}_{O2K} = ^{\mathcal W}\mathbf T_2 \  ^{\mathcal T_2}\bo{r}_{O2K}
			 \\
			 ^{\mathcal W} \bo{r}_{O2K} = ^{\mathcal W}\mathbf T_1 \ ^{\mathcal T_1}\mathbf T_2 \  ^{\mathcal T_2}\bo{r}_{O2K}
		\end{align}
		
		Finally, we observe that $\bo{r}_{O2K} = K - O_2$, $\bo{r}_{O2} = O_2 - O$ and $\bo{r}_{K} = K - O$, so:
		
		\begin{align}
			\bo{r}_{K} = \bo{r}_{O2} + \bo{r}_{O2K} 
			\\
			 ^{\mathcal W} \bo{r}_{K} =  ^{\mathcal W} \bo{r}_{O2} + ^{\mathcal W}  \bo{r}_{O2K} 
			 \\
			 ^{\mathcal W} \bo{r}_{K} =  
			 ^{\mathcal W}\bo{r}_{O1} + 
			 ^{\mathcal W}\mathbf T_1  \  ^{\mathcal T_1}\bo{r}_{O1O2}  +
			 ^{\mathcal W}\mathbf T_1 \ ^{\mathcal T_1}\mathbf T_2 \  ^{\mathcal T_2}\bo{r}_{O2K}
		\end{align}		
		
	\end{flushleft}
\end{frame}




\begin{frame}{Serial linkages}
	%\framesubtitle{1st order}
	\begin{flushleft}
		
		Consider a serial non-branching linkage. The $i$-th link has frame $\mathcal F_i$ attached to it, meaning the coordinates of any point on the link with respect to the frame do not change.
		
		\bigskip
		
		For simplicity, let us assume that the joint ($i$-th) by which the $i$-th link is attached (to its \emph{parent link}) has coordinates $[0, \ 0, \ 0]$ in the frame $\mathcal F_i$. The point $O_i$ is $i$-th joint and the origin of the frame $\mathcal F_i$; the basis associated with that frame is $\mathcal T_i$. We define vector $\bo{r}_{O_i O_{i+1}}$ pointing from the current joint to the next one. For the last $n$-th link we define $\bo{r}_{O_n K}$ vector pointing from the joint to the end-effector
		
		\bigskip
		
		We know $^{\mathcal T_i}\bo{r}_{O_i O_{i+1}}$ and $^{\mathcal T_i}\bo{T}_{i+1}$. We need to find everything in terms of world frame.
		
	\end{flushleft}
\end{frame}



\begin{frame}{Serial linkages}
	%\framesubtitle{1st order}
	\begin{flushleft}
		
		The process is rather simple. First consider rotations. We are given $^{\mathcal W}\bo{T}_1$, $^{\mathcal T_1}\bo{T}_2$, $^{\mathcal T_2}\bo{T}_3$, etc.
		
		\bigskip
		
		We can use our familiar arguments about coordinates to show that:
		
		\begin{align}
			^{\mathcal W} \bo{T}_2 = 	^{\mathcal W} \bo{T}_1  \ ^{\mathcal T_1}\bo{T}_2 \\
			^{\mathcal W} \bo{T}_3 = 	^{\mathcal W} \bo{T}_2 \ ^{\mathcal T_2}\bo{T}_3\\
			^{\mathcal W} \bo{T}_{i+1} = 	^{\mathcal W} \bo{T}_i \ ^{\mathcal T_i}\bo{T}_{i+1}
		\end{align}
	
	We find \emph{absolute orientation} of a link $^{\mathcal W} \bo{T}_i$ and then we can find absolute orientation of the next link just by using its local basis  $^{\mathcal T_i}\bo{T}_{i+1}$.
	
	You can think about it as "we know orientation of the previous link - next one needs just one more rotation, given by $^{\mathcal T_i}\bo{T}_{i+1}$".
		
	\end{flushleft}
\end{frame}



\begin{frame}{Serial linkages}
	%\framesubtitle{1st order}
	\begin{flushleft}
		
		Next, we take "translations". All vectors $^{\mathcal T_i}\bo{r}_{O_i O_{i+1}}$ are known to us in local coordinates. We can transfer then to world coordinates:
		
		\begin{equation}
			^{\mathcal W}\bo{r}_{O_i O_{i+1}} = ^{\mathcal W} \bo{T}_i \  ^{\mathcal T_i}\bo{r}_{O_i O_{i+1}}
		\end{equation}
	
	With that, we can easily find position of any point on the robot, end-effector for instance:
	
		\begin{equation}
	^{\mathcal W}\bo{r}_K = 
	^{\mathcal W}\bo{r}_{O_1} +
	^{\mathcal W}\bo{r}_{O_1 O_2} +
	^{\mathcal W}\bo{r}_{O_2 O_3} + ... +
	^{\mathcal W}\bo{r}_{O_n K}
		\end{equation}	
		
	\end{flushleft}
\end{frame}




\begin{frame}{Rotation joints}
	%\framesubtitle{1st order}
	\begin{flushleft}
		
		How do we integrate joints into this discussion?
		
		\bigskip
		
		Remember rotation matrices:
		%
		\begin{equation}
			\bo{R}_x (\varphi_x) = 
			\begin{bmatrix}
				1 & 0 & 0 \\
				0 & \cos \varphi_x & -\sin \varphi_x \\
				0 & \sin \varphi_x & \cos \varphi_x 
			\end{bmatrix}
		\end{equation}
	
	\begin{equation}
	\bo{R}_y (\varphi_y) = 
	\begin{bmatrix}
		\cos \varphi_y & 0 & \sin \varphi_y \\
		0 & 1 & 0 \\
		-\sin \varphi_y & 0 & \cos \varphi_y
	\end{bmatrix}
\end{equation}

\begin{equation}
	\bo{R}_z (\varphi_z) = 
	\begin{bmatrix}
		\cos \varphi_z & -\sin \varphi_z & 0 \\
		\sin \varphi_z & \cos \varphi_z & 0 \\
		0 & 0 & 1
	\end{bmatrix}
\end{equation}

If your joint aligns with one of the principle axes of your parent basis, the transformation from the patent basis to the child basis can be defined by one of those matrices. That is quite common in practice.
		
	\end{flushleft}
\end{frame}



\begin{frame}{Rotation joints}
	%\framesubtitle{1st order}
	\begin{flushleft}
		
		Assume the $i$-th link is connected to the $i-1$ link via a pin joint aligned with the x-axes of the frame $\mathcal F_{i-1}$. Then, using our previous notation we get $ ^{\mathcal T_{i-1}}\bo{T}_i = \bo{R}_x (\varphi_i)$. We call it \emph{parametrization}. In this example $\varphi_i$ is a joint angle.
		
		\bigskip
		
		It is also easy to find examples where  $ ^{\mathcal T_{i-1}}\bo{T}_i = ^{\mathcal T_{i-1}}\bo{C}_i \ \bo{R}_x (\varphi_i)$ where $^{\mathcal T_{i-1}}\bo{C}_i = const$ that depicts how the child link is oriented when $\varphi_i = 0$. We can use this constant orientation offset to change pose of the robot corresponding to zero joint angles, as well as to account for geometric quirks of the linkage.
		
	\end{flushleft}
\end{frame}



\begin{frame}{Rotation joints}
	%\framesubtitle{1st order}
	\begin{flushleft}
		
		Let's draw examples!
		
	\end{flushleft}
\end{frame}




\begin{frame}{Thank you!}
\centerline{Lecture slides are available via Moodle.}
\bigskip
\centerline{You can help improve these slides at:}
\centerline{\mygit}
\bigskip
\centerline{Check Moodle for additional links, videos, textbook suggestions.}
\bigskip

\centerline{\textcolor{black}{\qrcode[height=1.6in]{https://github.com/SergeiSa/Fundumentals-of-robotics-2022}}}

\end{frame}

\end{document}
