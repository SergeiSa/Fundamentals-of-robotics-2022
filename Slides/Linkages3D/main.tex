\documentclass{beamer}

\input{settings.tex}


\title{3D linkages}
\subtitle{Fundamentals of Robotics, Lecture 2}
\author{by Sergei Savin}
\centering
\date{\mydate}



\begin{document}
\maketitle


%\begin{frame}{Content}
%
%%\begin{itemize}
%%\item Motivation
%%\item Ordinary differential equations
%%\item Linear differential equations
%%\item Changing n-th order ODE to a State-Space form
%%\item State-Space to ODE
%%\item Read more
%%\end{itemize}
%
%\end{frame}



\begin{frame}{Basis}
	%\framesubtitle{1st order}
	\begin{flushleft}
		
		
		Three linearly independent vectors $\bo{v}_1$, $\bo{v}_2$ and $\bo{v}_3$ in $\R^3$ form a \emph{basis} $\mathcal V$ - which means any vector in $\R^3$ can be represented as a linear combination of $\bo{v}_1$, $\bo{v}_2$ and $\bo{v}_3$.
		
		\bigskip
		
		If  $||\bo{v}_i|| = 1$ and  $\bo{v}_1 \cdot \bo{v}_2 = 0$, $\bo{v}_2 \cdot \bo{v}_3 = 0$, $\bo{v}_3 \cdot \bo{v}_1 = 0$, then the basis is called orthonormal.
		
		\bigskip
		
		If $\bo{V} = [\bo{v}_1 \ \bo{v}_2 \ \bo{v}_3]$ is an orthonormal basis, then:
		
		\begin{equation}
			\bo{V}\T \bo{V} = \bo{I} 
		\end{equation}
	
		Given a vector $\bo{r}$ we can find its coordinates in terms of  $ \bo{V}$ as:
		
		\begin{equation}
			^\mathcal{V} \bo{r} =\bo{V}\T \bo{r}
		\end{equation}		
	
		Note the abuse of notation: we talk interchangeably of about the basis $\mathcal V$ and matrix whose columns are the elements of the basis $\mathbf V$.
		
	\end{flushleft}
\end{frame}



\begin{frame}{World}
	%\framesubtitle{1st order}
	\begin{flushleft}
		
		
		Consider a basis $\mathcal{V}$ defined via matrix $\bo{V} = [\bo{v}_1 \ \bo{v}_2 \ \bo{v}_3]$. If vectors $\bo{v}_i$ are actually given by their coordinates, we can consider these vectors as being expressed in a \emph{world} basis $\mathcal{W}$.
		
		\bigskip
		
		Given a vector $\bo{r}$, we can express it in terms of basis $\mathcal{V}$, denoting it with a left upper-script:  $^\mathcal{V} \bo{r}$. But we can also express it in the world basis as $^\mathcal{W} \bo{r}$. Relation between  $^\mathcal{V} \bo{r}$ and  $^\mathcal{W} \bo{r}$ is given as:
		
		\begin{equation}
			 ^\mathcal{V} \bo{r} = \bo{V}\T  \ ^\mathcal{W} \bo{r} 
		\end{equation}
		
		
	\end{flushleft}
\end{frame}




\begin{frame}{Frame}
	%\framesubtitle{1st order}
	\begin{flushleft}
		
		An orthonormal basis $\bo{V} = [\bo{v}_1 \ \bo{v}_2 \ \bo{v}_3]$ and an origin point $ \bo{r}$ together form a frame $\mathcal{F}$. Usually we assume that $\bo{v}_3 = \bo{v}_1 \times \bo{v}_2$, $\bo{v}_1 = \bo{v}_2 \times \bo{v}_3$ and $\bo{v}_2 = \bo{v}_3 \times \bo{v}_1$.
		
		\bigskip
		
		Given a point $\bo{p}$ we can find its coordinates in the frame $\mathcal{F}$:
		
		\begin{equation}
			^\mathcal{F} \bo{p} =\bo{V}\T (\bo{p} - \bo{r})
		\end{equation}
	
		\bigskip
	
		A few notes:
		
		\begin{itemize}
			\item Vectors are usually expressed in terms of bases, points are usually expressed in terms of frames.
			
			\item Any point can be associated with a vector with the same coordinates, which can be expressed in a basis.
		\end{itemize}
		
		
	\end{flushleft}
\end{frame}




\begin{frame}{Geometry of a link}
	%\framesubtitle{1st order}
	\begin{flushleft}
		
		
		Consider a rigid body. We mark three points on that body: $C$, $B$ and $F$: center of mass, base and follower. We also define a basis $\mathcal{B}$ stationary with respect to the body $\mathcal{B} = [\bo{b}_x, \ \bo{b}_y, \ \bo{b}_z]$ and coordinates of these points in $\mathcal{B}$: $\bo{r}_C$, $\bo{r}_B$ and $\bo{r}_F$.
		
		\bigskip
		
		Considering vectors $\bo{b}_x$, $\bo{b}_y$ and $\bo{b}_z$ as expressed in terms of some default basis $\bo{I}$, we can find coordinates 
		
		
		
	\end{flushleft}
\end{frame}



\begin{frame}{One link robot}
	%\framesubtitle{1st order}
	\begin{flushleft}
		
		Let us consider a pendulum, described with angle $\varphi$ between its link and the vertical line. The length of the pendulum is given as $l$, point $\bo{r}_K$ is located at end of the pendulum.
		
		\bigskip
		
		\begin{columns}
			\begin{column}{0.5\textwidth}
				\tikzset{every picture/.style={line width=0.75pt}} %set default line width to 0.75pt        

\begin{tikzpicture}[x=0.75pt,y=0.75pt,yscale=-1,xscale=1]
	%uncomment if require: \path (0,300); %set diagram left start at 0, and has height of 300
	
	%Straight Lines [id:da1514141660071826] 
	\draw [color={rgb, 255:red, 155; green, 155; blue, 155 }  ,draw opacity=1 ]   (171,175) -- (171,39) ;
	\draw [shift={(171,37)}, rotate = 90] [color={rgb, 255:red, 155; green, 155; blue, 155 }  ,draw opacity=1 ][line width=0.75]    (10.93,-3.29) .. controls (6.95,-1.4) and (3.31,-0.3) .. (0,0) .. controls (3.31,0.3) and (6.95,1.4) .. (10.93,3.29)   ;
	%Straight Lines [id:da8867696028247825] 
	\draw [color={rgb, 255:red, 155; green, 155; blue, 155 }  ,draw opacity=1 ]   (92,69) -- (237,70.97) ;
	\draw [shift={(239,71)}, rotate = 180.78] [color={rgb, 255:red, 155; green, 155; blue, 155 }  ,draw opacity=1 ][line width=0.75]    (10.93,-3.29) .. controls (6.95,-1.4) and (3.31,-0.3) .. (0,0) .. controls (3.31,0.3) and (6.95,1.4) .. (10.93,3.29)   ;
	%Straight Lines [id:da86988252841081] 
	\draw    (171,69) -- (213,170) ;
	%Shape: Circle [id:dp42453397326631825] 
	\draw  [fill={rgb, 255:red, 0; green, 0; blue, 0 }  ,fill opacity=1 ] (209.5,170) .. controls (209.5,168.07) and (211.07,166.5) .. (213,166.5) .. controls (214.93,166.5) and (216.5,168.07) .. (216.5,170) .. controls (216.5,171.93) and (214.93,173.5) .. (213,173.5) .. controls (211.07,173.5) and (209.5,171.93) .. (209.5,170) -- cycle ;
	%Shape: Pie [id:dp2450860928862184] 
	\draw  [color={rgb, 255:red, 155; green, 155; blue, 155 }  ,draw opacity=1 ] (184.19,100.45) .. controls (180.21,102.72) and (175.73,104) .. (171,104) -- (171,69) -- cycle ;
	%Shape: Circle [id:dp7741237432120269] 
	\draw  [fill={rgb, 255:red, 255; green, 255; blue, 255 }  ,fill opacity=1 ] (164,69) .. controls (164,65.13) and (167.13,62) .. (171,62) .. controls (174.87,62) and (178,65.13) .. (178,69) .. controls (178,72.87) and (174.87,76) .. (171,76) .. controls (167.13,76) and (164,72.87) .. (164,69) -- cycle ;
	
	% Text Node
	\draw (244,49.73) node [anchor=north west][inner sep=0.75pt]  [color={rgb, 255:red, 155; green, 155; blue, 155 }  ,opacity=1 ]  {$Ox$};
	% Text Node
	\draw (143,14.73) node [anchor=north west][inner sep=0.75pt]  [color={rgb, 255:red, 155; green, 155; blue, 155 }  ,opacity=1 ]  {$Oy$};
	% Text Node
	\draw (223,144.4) node [anchor=north west][inner sep=0.75pt]  [color={rgb, 255:red, 65; green, 117; blue, 5 }  ,opacity=1 ]  {$\bo{r}_{K}$};
	% Text Node
	\draw (179,46.4) node [anchor=north west][inner sep=0.75pt]  [color={rgb, 255:red, 65; green, 117; blue, 5 }  ,opacity=1 ]  {$\bo{r}_{O}$};
	% Text Node
	\draw (173,107.4) node [anchor=north west][inner sep=0.75pt]  [color={rgb, 255:red, 65; green, 117; blue, 5 }  ,opacity=1 ]  {$\varphi $};
	
	
\end{tikzpicture}

			\end{column}
			\begin{column}{0.5\textwidth}  %%<--- here
				
				Position of $\bo{r}_K$ is given as:
				
				\begin{equation}
					\bo{r}_K = \begin{bmatrix}
						l \sin (\varphi) \\ 
						-l \cos (\varphi) 
					\end{bmatrix}
				\end{equation}
				
				Differentiating the position we can find its velocity:
				
				\begin{equation}
					\dot{\bo{r}}_K = \begin{bmatrix}
						 l \dot \varphi \cos (\varphi) \\ 
						 l \dot \varphi \sin (\varphi) 
					\end{bmatrix}
				\end{equation}
				
				
			\end{column}
		\end{columns}
		
		
	\end{flushleft}
\end{frame}



\begin{frame}{One link robot}
	%\framesubtitle{1st order}
	\begin{flushleft}
		
		The forward kinematics can also be solved via rotation matrices:
		
		\begin{equation}
			\bo{r}_K = 
			\ma{\cos(\varphi)}{-\sin (\varphi)}{\sin (\varphi)}{\cos(\varphi)}
			\begin{bmatrix}
				0 \\ 
				-l 
			\end{bmatrix}
		=
			\begin{bmatrix}
				l \sin (\varphi) \\ 
				-l \cos (\varphi) 
			\end{bmatrix}
		\end{equation}
		
		
		
		
	\end{flushleft}
\end{frame}



\begin{frame}{Two link robot}
	%\framesubtitle{1st order}
	\begin{flushleft}
		
	
		\begin{columns}
			\begin{column}{0.4\textwidth}
				


\tikzset{every picture/.style={line width=0.75pt}} %set default line width to 0.75pt        

\begin{tikzpicture}[x=0.75pt,y=0.75pt,yscale=-1,xscale=1]
	%uncomment if require: \path (0,300); %set diagram left start at 0, and has height of 300
	
	%Shape: Pie [id:dp044024582329474704] 
	\draw  [color={rgb, 255:red, 155; green, 155; blue, 155 }  ,draw opacity=1 ] (237.35,190.45) .. controls (234.52,195.02) and (230.77,198.77) .. (226.42,201.31) -- (213,170) -- cycle ;
	%Shape: Pie [id:dp8327443137283965] 
	\draw  [color={rgb, 255:red, 155; green, 155; blue, 155 }  ,draw opacity=1 ] (184.19,100.45) .. controls (180.21,102.72) and (175.73,104) .. (171,104) -- (171,69) -- cycle ;
	%Straight Lines [id:da11981673752029831] 
	\draw [color={rgb, 255:red, 155; green, 155; blue, 155 }  ,draw opacity=1 ]   (171,175) -- (171,39) ;
	\draw [shift={(171,37)}, rotate = 90] [color={rgb, 255:red, 155; green, 155; blue, 155 }  ,draw opacity=1 ][line width=0.75]    (10.93,-3.29) .. controls (6.95,-1.4) and (3.31,-0.3) .. (0,0) .. controls (3.31,0.3) and (6.95,1.4) .. (10.93,3.29)   ;
	%Straight Lines [id:da4019711414513023] 
	\draw [color={rgb, 255:red, 155; green, 155; blue, 155 }  ,draw opacity=1 ]   (92,69) -- (237,70.97) ;
	\draw [shift={(239,71)}, rotate = 180.78] [color={rgb, 255:red, 155; green, 155; blue, 155 }  ,draw opacity=1 ][line width=0.75]    (10.93,-3.29) .. controls (6.95,-1.4) and (3.31,-0.3) .. (0,0) .. controls (3.31,0.3) and (6.95,1.4) .. (10.93,3.29)   ;
	%Straight Lines [id:da26868524985077347] 
	\draw    (171,69) -- (213,170) ;
	%Shape: Circle [id:dp05206742113524521] 
	\draw  [fill={rgb, 255:red, 0; green, 0; blue, 0 }  ,fill opacity=1 ] (287.75,233) .. controls (287.75,231.21) and (289.21,229.75) .. (291,229.75) .. controls (292.79,229.75) and (294.25,231.21) .. (294.25,233) .. controls (294.25,234.79) and (292.79,236.25) .. (291,236.25) .. controls (289.21,236.25) and (287.75,234.79) .. (287.75,233) -- cycle ;
	%Shape: Circle [id:dp009854061062177122] 
	\draw  [fill={rgb, 255:red, 255; green, 255; blue, 255 }  ,fill opacity=1 ] (164,69) .. controls (164,65.13) and (167.13,62) .. (171,62) .. controls (174.87,62) and (178,65.13) .. (178,69) .. controls (178,72.87) and (174.87,76) .. (171,76) .. controls (167.13,76) and (164,72.87) .. (164,69) -- cycle ;
	%Straight Lines [id:da271351498464834] 
	\draw    (213,170) -- (291,233) ;
	%Straight Lines [id:da15597835937675186] 
	\draw [color={rgb, 255:red, 155; green, 155; blue, 155 }  ,draw opacity=1 ] [dash pattern={on 4.5pt off 4.5pt}]  (213,170) -- (234,220) ;
	%Shape: Circle [id:dp48139269340211954] 
	\draw  [fill={rgb, 255:red, 255; green, 255; blue, 255 }  ,fill opacity=1 ] (206,170) .. controls (206,166.13) and (209.13,163) .. (213,163) .. controls (216.87,163) and (220,166.13) .. (220,170) .. controls (220,173.87) and (216.87,177) .. (213,177) .. controls (209.13,177) and (206,173.87) .. (206,170) -- cycle ;
	
	% Text Node
	\draw (244,49.73) node [anchor=north west][inner sep=0.75pt]  [color={rgb, 255:red, 155; green, 155; blue, 155 }  ,opacity=1 ]  {$Ox$};
	% Text Node
	\draw (143,14.73) node [anchor=north west][inner sep=0.75pt]  [color={rgb, 255:red, 155; green, 155; blue, 155 }  ,opacity=1 ]  {$Oy$};
	% Text Node
	\draw (301,220.4) node [anchor=north west][inner sep=0.75pt]  [color={rgb, 255:red, 65; green, 117; blue, 5 }  ,opacity=1 ]  {$\mathbf r_{K}$};
	% Text Node
	\draw (179,46.4) node [anchor=north west][inner sep=0.75pt]  [color={rgb, 255:red, 65; green, 117; blue, 5 }  ,opacity=1 ]  {$\mathbf r_{O1}$};
	% Text Node
	\draw (173,107.4) node [anchor=north west][inner sep=0.75pt]  [color={rgb, 255:red, 65; green, 117; blue, 5 }  ,opacity=1 ]  {$q_{1}$};
	% Text Node
	\draw (217,145.4) node [anchor=north west][inner sep=0.75pt]  [color={rgb, 255:red, 65; green, 117; blue, 5 }  ,opacity=1 ]  {$\mathbf r_{O2}$};
	% Text Node
	\draw (233,196.4) node [anchor=north west][inner sep=0.75pt]  [color={rgb, 255:red, 65; green, 117; blue, 5 }  ,opacity=1 ]  {$q_{2}$};
	
	
\end{tikzpicture}
			\end{column}
			\begin{column}{0.6\textwidth}  %%<--- here
				
					Let us consider a double-pendulum, described with angles $q_1$, $q_2$, with joints $O_1$, $O_2$ and \emph{end effector} $\bo{r}_K$, defined as before. The position of $K$ is given as:
				
					
					\begin{equation*}
						\bo{r}_K = \begin{bmatrix}
							l_1 \sin (q_1) + l_2 \sin (q_1+q_2) \\ 
							l_1 \cos (q_1) + l_2 \cos (q_1+q_2) 
						\end{bmatrix}
					\end{equation*}
				
			\end{column}
		\end{columns}
		
		
	\end{flushleft}
\end{frame}



\begin{frame}{Two link robot}
	%\framesubtitle{1st order}
	\begin{flushleft}
		
		Velocity of the end effector is found by differentiation:
		
				\begin{equation}
					\dot{\bo{r}}_K = 
					\begin{bmatrix}
						l_1 \dot q_1\cos (q_1) + l_2 (\dot q_1+\dot q_2) \cos (q_1+q_2) \\ 
					  - l_1 \dot q_1\sin (q_1)  - l_2 (\dot q_1+\dot q_2) \sin (q_1+q_2) 
					\end{bmatrix}
				\end{equation}
		
		If we introduce different variables $\varphi_1 = q_1$ and $\varphi_2 = q_1 + q_2$, the position and velocity of the end effector will take a simpler form:
		
		\begin{equation}
			\bo{r}_K = \begin{bmatrix}
				l_1 \sin (\varphi_1) + l_2 \sin (\varphi_2) \\ 
				l_1 \cos (\varphi_1) + l_2 \cos (\varphi_2) 
			\end{bmatrix}
		\end{equation}
		
		\begin{equation}
			\dot{\bo{r}}_K = 
			\begin{bmatrix}
				l_1 \dot \varphi_1\cos (\varphi_1) + l_2 \dot \varphi_2 \cos (\varphi_2) \\ 
			  - l_1 \dot \varphi_1\sin (\varphi_1)  - l_2 \dot \varphi_2 \sin (\varphi_2) 
			\end{bmatrix}
		\end{equation}
		
		\emph{The form of the kinematics depends on the choice of coordinates}. In this example $q_1$, $q_2$ are relative (joint) angles, and $\varphi_1$, $\varphi_2$ are absolute orientation.
		
	\end{flushleft}
\end{frame}



\begin{frame}{Two link robot}
	%\framesubtitle{1st order}
	\begin{flushleft}
		
		The forward kinematics can also be solved via rotation matrices:
		
		$$
			\bo{r}_K = 
			\ma{\cos(q_1)}{-\sin (q_1)}{\sin (q_1)}{\cos(q_1)}
			\begin{bmatrix}
				0 \\ 
				-l_1 
			\end{bmatrix}
		+$$
		$$+
		\ma{\cos(q_1+q_2)}{-\sin (q_1+q_2)}{\sin (q_1+q_2)}{\cos(q_1+q_2)}
		\begin{bmatrix}
			0 \\ 
			-l_2 
		\end{bmatrix}
			=$$
			$$
			=
			\begin{bmatrix}
				l_1 \dot q_1\cos (q_1) + l_2 (\dot q_1+\dot q_2) \cos (q_1+q_2) \\ 
				- l_1 \dot q_1\sin (q_1)  - l_2 (\dot q_1+\dot q_2) \sin (q_1+q_2) 
			\end{bmatrix}
		$$
		
		where:
		
		$$ \ma{\cos(q_1+q_2)}{-\sin (q_1+q_2)}{\sin (q_1+q_2)}{\cos(q_1+q_2)}  =$$
		$$=
		\ma{\cos(q_1)}{-\sin (q_1)}{\sin (q_1)}{\cos(q_1)}
		\ma{\cos(q_2)}{-\sin (q_2)}{\sin (q_2)}{\cos(q_2)}
		$$ 
		
		
		
		
		
	\end{flushleft}
\end{frame}



\begin{frame}{Angular velocity}
	%\framesubtitle{1st order}
	\begin{flushleft}
		
		Angular velocity of a rigid body is the rate of change of its orientation. In the last example, the orientation of the first link is given as $q_1$ and orientation of the second body - as $q_1 + q_2$. The rate of change of the orientation is found by differentiating these quantities:
		
		\begin{equation}
			\omega_1 = \dot q_1
		\end{equation}
		\begin{equation}
			\omega_2 = \dot q_1 + \dot q_2  
		\end{equation}
		
		This is so simple because we have an explicit expression defining orientation of the body; we will see next lecture that it is more difficult in 3D.
		
	\end{flushleft}
\end{frame}




\begin{frame}{Thank you!}
\centerline{Lecture slides are available via Moodle.}
\bigskip
\centerline{You can help improve these slides at:}
\centerline{\mygit}
\bigskip
\centerline{Check Moodle for additional links, videos, textbook suggestions.}
\bigskip

\centerline{\textcolor{black}{\qrcode[height=1.6in]{https://github.com/SergeiSa/Fundumentals-of-robotics-2022}}}

\end{frame}

\end{document}
