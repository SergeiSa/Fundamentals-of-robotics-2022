\documentclass{beamer}

\input{settings.tex}

\newcommand{\dq} {\dot{\mathbf{q}}}


\title{Task Prioritization}
\subtitle{Fundamentals of Robotics, Lecture 7}
\author{by Sergei Savin}
\centering
\date{\mydate}



\begin{document}
\maketitle





\begin{frame}{Linear Velocity and Angular Velocity}
	\begin{flushleft}
		
		Consider a point $K$ (position is given by vector $\bo{r}_K(\bo{q})$). We can formulate the velocity problem as such - find such $\dot{\bo{q}}^*$ that:
		
		\begin{equation}
			\bo{J}_K \dot{\bo{q}}^* = \bo{v}_K
		\end{equation}		
		
		We can solve velocity problem for the point $K$:
		
		\begin{equation}
			\dot{\bo{q}}^* = \bo{J}_K^+ \bo{v}_K + \bo{N} \bo{z}
		\end{equation}
	
		Now, assume that we want to also find such velocity that one of the links acquires angular velocity $\omega$:
	
		\begin{equation}
			\bo{J}_\omega \dot{\bo{q}}^* = \omega
		\end{equation}							
		
	\end{flushleft}
\end{frame}




\begin{frame}{Linear Velocity and Angular Velocity}
	\begin{flushleft}
		
		How do we solve it? We can easily combine both problems together and then we get:
		
		\begin{equation}
			\begin{bmatrix}
				\bo{J}_K \\ \bo{J}_\omega
			\end{bmatrix}
			 \dot{\bo{q}}^* = 
			 \begin{bmatrix}
			 	\bo{v}_K \\ \omega
			 \end{bmatrix}
		\end{equation}				
	
		Which is solved as:
		
		\begin{equation}
	\dot{\bo{q}}^* = 
	\begin{bmatrix}
		\bo{J}_K \\ \bo{J}_\omega
	\end{bmatrix}^+
	\begin{bmatrix}
		\bo{v}_K \\ \omega
	\end{bmatrix}
\end{equation}			
		
		If a solution exists, we will obtain it. But if a solution does not exit, we will find a least-residual solution, with errors distributed across both \emph{tasks}. What if we want to make sure one task is achieved, while the second task can be failed?
		
	\end{flushleft}
\end{frame}




\begin{frame}{Linear Velocity and Angular Velocity}
	\begin{flushleft}
		
		As we noted before, all solutions to the velocity problem are given as:
		
		\begin{equation}
			\dot{\bo{q}}^* = \bo{J}_K^+ \bo{v}_K + \bo{N} \bo{z}
		\end{equation}
		
		Then we can re-write the angular velocity problem $\bo{J}_\omega \dot{\bo{q}}^* = \omega$ as:
		
		\begin{equation}
			\bo{J}_\omega (\bo{J}_K^+ \bo{v}_K + \bo{N} \bo{z}) = \omega
		\end{equation}		
	
	This can be solved:
	
		\begin{align}
			\bo{J}_\omega \bo{J}_K^+ \bo{v}_K + \bo{J}_\omega \bo{N} \bo{z} = \omega \\
			\bo{z} = (\bo{J}_\omega\bo{N} )^{-1}(\omega - \bo{J}_\omega \bo{J}_K^+ \bo{v}_K) \\
			\dot{\bo{q}}^{**} =  \bo{N}(\bo{J}_\omega\bo{N} )^{-1}(\omega - \bo{J}_\omega \bo{J}_K^+ \bo{v}_K) + \bo{J}_K^+ \bo{v}_K
		\end{align}			
		
	\end{flushleft}
\end{frame}



\begin{frame}{Linear Velocity and Angular Velocity}
	\begin{flushleft}
		
		Considering the solution $\dot{\bo{q}}^{**} =  \bo{N}(\bo{J}_\omega\bo{N} )^{-1}(\omega - \bo{J}_\omega \bo{J}_K^+ \bo{v}_K) + \bo{J}_K^+ \bo{v}_K$ we can observe the following:
		
		\bigskip
		
		\begin{itemize}
			\item The solution looks a little ugly.
			
			\item It looks like it will only get worse if we consider the third task.
		\end{itemize}
	
	\bigskip
	
		We can do better.
		
	\end{flushleft}
\end{frame}



\begin{frame}{Task prioritization}
	\begin{flushleft}
		
		By now we know how to solve a problem of the type $\bo{J}_1 \dq =\bo{v}_1$. The solution is:
		
		\begin{equation}
			\dq_1 = \bo{J}_1^+ \bo{v}_1
		\end{equation}
	
		We add second task $\bo{J}_2 (\dq_1 + \dq_2) =\bo{v}_2$. Can we solve it, while keeping the solution to the first task $\bo{J}_1 (\dq_1 + \dq_2) =\bo{v}_1$?
		
		\bigskip
		
		Our proposition: we solve an alternative task:
		
		\begin{align}
			\bo{J}_2 \bo{P}_1 (\dq_1 + \dq_2) =\bo{v}_2 \\
			\bo{P}_1 = \bo{I} - \bo{J}_1^+ \bo{J}_1
		\end{align}
		
		We claim that the first and the second task will both be satisfied (if possible) even if we solve them \emph{sequentially}.
		
	\end{flushleft}
\end{frame}



\begin{frame}{Task prioritization}
	\begin{flushleft}
		
		Let us study the proposed solution. First, the joint velocities $\dq_1$ are found as $\dq_1 = \bo{J}_1^+ \bo{v}_1$.
		
		\bigskip
		
		Second, matrix $\bo{P}_1 = \bo{I} - \bo{J}_1^+ \bo{J}_1$ is a null space projector for the jacobian $\bo{J}_1$. 
		
		\bigskip
		
		Third, we consider equation $\bo{J}_2 \bo{P}_1 (\dq_1 + \dq_2) =\bo{v}_2$:
		
		\begin{equation}
			\bo{J}_2 \bo{P}_1 (\bo{J}_1^+ \bo{v}_1 + \dq_2) =\bo{v}_2
		\end{equation}
	
		Since $\bo{J}_1^+ \in \text{row}(\bo{J}_1)$ we can conclude that $\bo{P}_1 \bo{J}_1^+ = 0$. Finally, we have the equation in the form:
		
		\begin{equation}
			\bo{J}_2 \bo{P}_1 \dq_2 =\bo{v}_2
		\end{equation}		
		
	\end{flushleft}
\end{frame}



\begin{frame}{Task prioritization}
	\begin{flushleft}
		
		Given $\bo{J}_2 \bo{P}_1 \dq_2 =\bo{v}_2$ we can solve it:
		
		\begin{align}
			\bo{P}_1 \dq_2 =\bo{J}_2^+ \bo{v}_2 \\
			\bo{P}_1 = \bo{C}_1 \bo{C}_1\T \\
		    \bo{C}_1 \bo{C}_1\T \dq_2 =\bo{J}_2^+ \bo{v}_2 \\
			 \bo{C}_1\T \dq_2 =\bo{C}_1\T \bo{J}_2^+ \bo{v}_2 \\
			 \dq_2 = \bo{C}_1 \bo{C}_1\T \bo{J}_2^+ \bo{v}_2 \\
			 \dq_2 = \bo{P}_1 \bo{J}_2^+ \bo{v}_2
		\end{align}		
	
		We can summarize it as $(\bo{J}_2 \bo{P}_1)^+  = \bo{P}_1 \bo{J}_2^+$.
		
	\end{flushleft}
\end{frame}



\begin{frame}{Task prioritization - verification}
	\begin{flushleft}
		
		So, The solution is: 
		
		\begin{align}
			\dq_1 = \bo{J}_1^+ \bo{v}_1 \\
			\dq_2 = \bo{P}_1 \bo{J}_2^+ \bo{v}_2
		\end{align}	
	
		Can we prove that the following holds?
		
		\begin{equation}
			\begin{cases}
				\bo{J}_1 (\dq_1 + \dq_2) =\bo{v}_1 \\
				\bo{J}_2 \bo{P}_1 (\dq_1 + \dq_2) =\bo{v}_2
			\end{cases}
		\end{equation}			
		
	\end{flushleft}
\end{frame}



\begin{frame}{Task prioritization - verification}
	\begin{flushleft}
		
	 We study equation $\bo{J}_1 (\dq_1 + \dq_2) =\bo{v}_1$:
	 
	 	\begin{align}
	 		 \bo{J}_1\dq_1 + 
	 		 \bo{J}_1\bo{P}_1 \bo{J}_2^+ \bo{v}_2 =
	 		 \bo{v}_1
	 	\end{align}	
 	
 	Assuming that $\bo{J}_1\dq_1 = \bo{v}_1$ (the residual is zero), we get:
 	
	 	\begin{align}
	\bo{J}_1\bo{P}_1 \bo{J}_2^+ \bo{v}_2 = 0
		\end{align}	 	
		
		Since $\bo{P}_1$ is null space projector it means that its columns lie in the null space of $\bo{J}_1$ - meaning that $\bo{J}_1\bo{P}_1 = 0$, q.e.d.
		
	\end{flushleft}
\end{frame}


\begin{frame}{Task prioritization - verification}
	\begin{flushleft}
		
		We study equation $\bo{J}_2 \bo{P}_1 (\dq_1 + \dq_2) =\bo{v}_2$:
		
		\begin{align}
			\bo{J}_2 \bo{P}_1 \bo{J}_1^+ \bo{v}_1 + 
			\bo{J}_2 \bo{P}_1 \dq_2 =
			\bo{v}_2
		\end{align}	
	
%		For projectors it is true that $\bo{P}_1 = \bo{P}_1 \bo{P}_1$, so:
%		
%		\begin{align}
%	\bo{J}_2 \bo{J}_1^+ \bo{v}_1 + 
%	\bo{J}_2 \bo{P}_1 \bo{P}_1 \bo{J}_2^+ \bo{v}_2 =
%	\bo{v}_2 
%	\\
%	\bo{J}_2 \bo{J}_1^+ \bo{v}_1 + 
%	\bo{J}_2 \bo{P}_1  \dq_2 =
%	\bo{v}_2
%		\end{align}			
		
		If $\bo{J}_2 \bo{P}_1  \dq_2 =
		\bo{v}_2$, meaning that we found a zero-residual solution, then:
		
		\begin{align}
			\bo{J}_2 \bo{P}_1 \bo{J}_1^+ \bo{v}_1  = 0
		\end{align}	
	
	Since $\bo{J}_1^+$ is in the row space of $\bo{J}_1$, so $\bo{P}_1 \bo{J}_1^+ = 0$.
		
		\begin{align}
			\bo{J}_2 \bo{0} \bo{v}_1  &= 0 \\
			0 &= 0, \ \ \ \text{q.e.d.}
		\end{align}	
	
	Thus, we proved that the proposed solution works.
	
	\end{flushleft}
\end{frame}




\begin{frame}{3 tasks}
	\begin{flushleft}
		
		Let us try to solve three tasks, one after another:
		%
		\begin{align}
			\bo{J}_1 (\dq_1 + \dq_2 + \dq_3) = \bo{v}_1 \\
			\bo{J}_2 \bo{P}_1 (\dq_1 + \dq_2 + \dq_3) = \bo{v}_2 \\
			\bo{J}_3 \bo{P}_2 (\dq_1 + \dq_2 + \dq_3) = \bo{v}_3
		\end{align}
	%
		\begin{align}
			\bo{P}_1 =\bo{I} - \bo{J}_1^+\bo{J}_1 \\
			\bo{P}_2 =\bo{P}_1 (\bo{I} - \bo{J}_2^+\bo{J}_2)
		\end{align}
		
		Where $\bo{P}_1$ is a projector onto the null space of $\bo{J}_1$, and $\bo{P}_2$ is a projector onto the intersection of null spaces of $\bo{J}_1$ and $\bo{J}_2$.
		
		We propose to solve it as:
		%
		\begin{align}
	\dq_1 &= \bo{J}_1^+ \bo{v}_1 \\
	\dq_2 &= \bo{P}_1 \bo{J}_2^+ \bo{v}_2 \\
	\dq_3 &= \bo{P}_2 \bo{J}_3^+ \bo{v}_3
		\end{align}		
		
	\end{flushleft}
\end{frame}



\begin{frame}{3 tasks - verification}
	\begin{flushleft}
		
		First equation: $\bo{J}_1 (\dq_1 + \dq_2 + \dq_3) = \bo{v}_1$.
		
		\bigskip
		
		If the first task has a zero-residual $\bo{J}_1 \dq_1 = \bo{v}_1$, we obtain:
		
		\begin{align}
			\textcolor{mygray}{\text{(what we want to prove)}} \ \ \ 
			&\bo{J}_1 (\dq_2 + \dq_3) = 0 \\
			&\bo{J}_1 (\bo{P}_1 \bo{J}_2^+ \bo{v}_2 
			+ 
			\bo{P}_2 \bo{J}_3^+ \bo{v}_3) = 0 
		\end{align}
		
		We can observe $\bo{J}_1 \bo{P}_1 = 0$ and $\bo{J}_1 \bo{P}_2 =\bo{J}_1 \bo{P}_1 (\bo{I} - \bo{J}_2^+\bo{J}_2)  = 0$:
		
		\begin{align}
			\bo{J}_1 (\bo{P}_1 \bo{J}_2^+ \bo{v}_2 
			+ 
			\bo{P}_2 \bo{J}_3^+ \bo{v}_3) = \bo{J}_1(0 + 0) = 0
		\end{align}
		
	\end{flushleft}
\end{frame}




\begin{frame}{3 tasks - verification}
	\begin{flushleft}
		
		Second equation: $\bo{J}_2 \bo{P}_1 (\dq_1 + \dq_2 + \dq_3) = \bo{v}_2$.
		
		\bigskip
		
		Given that $\bo{J}_2 \bo{P}_1 \dq_2 = \bo{v}_2$, we obtain:
		%
		\begin{align}
			\textcolor{mygray}{\text{(what we want to prove)}} \ \ \ 
			\bo{J}_2 \bo{P}_1 (\dq_1 + \dq_3) = 0 
			\\
			\bo{J}_2 \bo{P}_1 (\bo{J}_1^+ \bo{v}_1 
			+ 
			\bo{P}_2 \bo{J}_3^+ \bo{v}_3) = 0 
			\\
			 \bo{J}_2 \bo{P}_1 \bo{P}_2 \bo{J}_3^+ \bo{v}_3 = 0  \\
			 \bo{J}_2 \bo{P}_2 \bo{J}_3^+ \bo{v}_3 = 0
		\end{align}
		
		Matrix $\bo{P}_2$ is null space projector for the $\bo{J}_2$, further projected onto the null space of $\bo{J}_1$; hence $\bo{J}_2 \bo{P}_2 = 0$: 
		
		\begin{align}
			\bo{J}_2 \bo{P}_2 \bo{J}_3^+ \bo{v}_3 = 
			\bo{0} \bo{J}_3^+ \bo{v}_3 = 0 = 0
		\end{align}
	
	\end{flushleft}
\end{frame}




\begin{frame}{3 tasks - verification}
	\begin{flushleft}
		
		Third equation: $\bo{J}_3 \bo{P}_2 (\dq_1 + \dq_2 + \dq_3) = \bo{v}_3$.
		
		\bigskip
		
		Given that $\bo{J}_3 \bo{P}_2 \dq_3 = \bo{v}_3$, we obtain:
		
		\begin{align}
			\textcolor{mygray}{\text{(what we want to prove)}} \ \ \ 
			\bo{J}_3 \bo{P}_2 (\dq_1 + \dq_2) = 0 
			\\
			\bo{J}_3 \bo{P}_2 (
			\bo{J}_1^+ \bo{v}_1
			 + 
			 \bo{P}_1 \bo{J}_2^+ \bo{v}_2) = 0 
		\end{align}
		
		We can observe that $\bo{P}_2  \bo{J}_1^+ = 0$ and $\bo{P}_2 \bo{P}_1 \bo{J}_2^+ = 0$, so
		
		\begin{align}
	\bo{J}_3 \bo{P}_2\bo{J}_1^+ \bo{v}_1
	+ 
	\bo{J}_3 \bo{P}_2 \bo{P}_1 \bo{J}_2^+ \bo{v}_2
	= 
	\bo{J}_3 \bo{0} \bo{v}_1
	+ 
	\bo{J}_3 \bo{0} \bo{v}_2  = 0 = 0
		\end{align}		
		
	\end{flushleft}
\end{frame}



\begin{frame}{N tasks - verification}
	\begin{flushleft}
		
		In general, we have the following method for sequential tasks:
		
		\begin{align}
			\dq_i &= \bo{P}_{i-1} \bo{J}_i^+ \bo{v}_i \\
			\bo{P}_i & =\bo{P}_{i-1} (\bo{I} - \bo{J}_i^+\bo{J}_i)
		\end{align}		
		
		We can see advantages of the approach:
		
		\begin{itemize}
			\item Complexity does not increase with the number of tasks
			
			\item We only need to invert jacobians once
		\end{itemize}
		
	\end{flushleft}
\end{frame}




\begin{frame}{Thank you!}
\centerline{Lecture slides are available via Moodle.}
\bigskip
\centerline{You can help improve these slides at:}
\centerline{\mygit}
\bigskip
\centerline{Check Moodle for additional links, videos, textbook suggestions.}
\bigskip

\centerline{\textcolor{black}{\qrcode[height=1.6in]{https://github.com/SergeiSa/Fundamentals-of-robotics-2022}}}

\end{frame}

\end{document}
