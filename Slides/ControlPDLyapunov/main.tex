\documentclass{beamer}

\pdfmapfile{+sansmathaccent.map}


\mode<presentation>
{
  \usetheme{Warsaw} % or try Darmstadt, Madrid, Warsaw, Rochester, CambridgeUS, ...
  \usecolortheme{seahorse} % or try seahorse, beaver, crane, wolverine, ...
  \usefonttheme{serif}  % or try serif, structurebold, ...
  \setbeamertemplate{navigation symbols}{}
  \setbeamertemplate{caption}[numbered]
} 


%%%%%%%%%%%%%%%%%%%%%%%%%%%%
% itemize settings

\definecolor{mypink}{RGB}{255, 30, 80}
\definecolor{mydarkblue}{RGB}{60, 160, 255}
\definecolor{mydarkred}{RGB}{160, 30, 30}
\definecolor{mylightred}{RGB}{255, 150, 150}
\definecolor{myred}{RGB}{200, 110, 110}
\definecolor{myblackblue}{RGB}{40, 40, 120}
\definecolor{myblackred}{RGB}{120, 40, 40}
\definecolor{myblue}{RGB}{240, 240, 255}
\definecolor{mygreen}{RGB}{0, 200, 0}
\definecolor{mygreen2}{RGB}{205, 255, 200}
\definecolor{mygray}{gray}{0.8}
% \definecolor{mydarkgray}{gray}{0.4}
\definecolor{mydarkgray}{RGB}{80, 80, 160}

\setbeamertemplate{itemize items}[default]

\setbeamertemplate{itemize item}{\color{myblackred}$\blacksquare$}
\setbeamertemplate{itemize subitem}{\color{mydarkred}$\blacktriangleright$}
\setbeamertemplate{itemize subsubitem}{\color{mygray}$\blacksquare$}

\setbeamercolor{palette quaternary}{fg=white,bg=myred} %mydarkgray
\setbeamercolor{titlelike}{parent=palette quaternary}

\setbeamercolor{palette quaternary2}{fg=white,bg=mydarkred}%black myblue
\setbeamercolor{frametitle}{parent=palette quaternary2}

\setbeamerfont{frametitle}{size=\Large,series=\scshape}
\setbeamerfont{framesubtitle}{size=\normalsize,series=\upshape}





%%%%%%%%%%%%%%%%%%%%%%%%%%%%
% block settings

\setbeamercolor{block title}{bg=red!30,fg=black}

\setbeamercolor*{block title example}{bg=mygreen!40!white,fg=black}

\setbeamercolor*{block body example}{fg= black, bg= mygreen2}


%%%%%%%%%%%%%%%%%%%%%%%%%%%%
% URL settings
\hypersetup{
    colorlinks=true,
    linkcolor=blue,
    filecolor=blue,      
    urlcolor=blue,
}

%%%%%%%%%%%%%%%%%%%%%%%%%%

\renewcommand{\familydefault}{\rmdefault}

\usepackage{amsmath}
\usepackage{mathtools}
\usepackage{mathrsfs}


\usepackage{subcaption}

\usepackage{qrcode}

\DeclareMathOperator*{\argmin}{arg\,min}
\newcommand{\bo}[1] {\mathbf{#1}}

\newcommand{\dx}[1] {\dot{\mathbf{#1}}}
\newcommand{\ma}[4] {\begin{bmatrix}
    #1 & #2 \\ #3 & #4
    \end{bmatrix}}
\newcommand{\myvec}[2] {\begin{bmatrix}
    #1 \\ #2
    \end{bmatrix}}
\newcommand{\myvecT}[2] {\begin{bmatrix}
    #1 & #2
    \end{bmatrix}}
 
 \newcommand{\R}{\mathbb{R}} 
 \newcommand{\T}{^\top}     
    

\newcommand{\mydate}{Fall 2022}
\newcommand{\mygit}{\textcolor{blue}{\href{https://github.com/SergeiSa/Fundamentals-of-robotics-2022}{github.com/SergeiSa/Fundamentals-of-robotics-2022}}}


\newcommand{\bref}[2] {\textcolor{blue}{\href{#1}{#2}}}




%%%%%%%%%%%%%%%%%%%%%%%%%%%%
% code settings

\usepackage{listings}
\usepackage{color}
% \definecolor{mygreen}{rgb}{0,0.6,0}
% \definecolor{mygray}{rgb}{0.5,0.5,0.5}
\definecolor{mymauve}{rgb}{0.58,0,0.82}
\lstset{ 
  backgroundcolor=\color{white},   % choose the background color; you must add \usepackage{color} or \usepackage{xcolor}; should come as last argument
  basicstyle=\footnotesize,        % the size of the fonts that are used for the code
  breakatwhitespace=false,         % sets if automatic breaks should only happen at whitespace
  breaklines=true,                 % sets automatic line breaking
  captionpos=b,                    % sets the caption-position to bottom
  commentstyle=\color{mygreen},    % comment style
  deletekeywords={...},            % if you want to delete keywords from the given language
  escapeinside={\%*}{*)},          % if you want to add LaTeX within your code
  extendedchars=true,              % lets you use non-ASCII characters; for 8-bits encodings only, does not work with UTF-8
  firstnumber=0000,                % start line enumeration with line 0000
  frame=single,	                   % adds a frame around the code
  keepspaces=true,                 % keeps spaces in text, useful for keeping indentation of code (possibly needs columns=flexible)
  keywordstyle=\color{blue},       % keyword style
  language=Octave,                 % the language of the code
  morekeywords={*,...},            % if you want to add more keywords to the set
  numbers=left,                    % where to put the line-numbers; possible values are (none, left, right)
  numbersep=5pt,                   % how far the line-numbers are from the code
  numberstyle=\tiny\color{mygray}, % the style that is used for the line-numbers
  rulecolor=\color{black},         % if not set, the frame-color may be changed on line-breaks within not-black text (e.g. comments (green here))
  showspaces=false,                % show spaces everywhere adding particular underscores; it overrides 'showstringspaces'
  showstringspaces=false,          % underline spaces within strings only
  showtabs=false,                  % show tabs within strings adding particular underscores
  stepnumber=2,                    % the step between two line-numbers. If it's 1, each line will be numbered
  stringstyle=\color{mymauve},     % string literal style
  tabsize=2,	                   % sets default tabsize to 2 spaces
  title=\lstname                   % show the filename of files included with \lstinputlisting; also try caption instead of title
}


%%%%%%%%%%%%%%%%%%%%%%%%%%%%
% URL settings
\hypersetup{
    colorlinks=false,
    linkcolor=blue,
    filecolor=blue,      
    urlcolor=blue,
}

%%%%%%%%%%%%%%%%%%%%%%%%%%

%%%%%%%%%%%%%%%%%%%%%%%%%%%%
% tikz settings

\usepackage{tikz}
\tikzset{every picture/.style={line width=0.75pt}}

\newcommand{\dq} {\dot{\mathbf{q}}}


\title{Nonlinear Control, PD, Gravity compensation, Lyapunov}
\subtitle{Fundamentals of Robotics, Lecture 9}
\author{by Sergei Savin}
\centering
\date{\mydate}



\begin{document}
\maketitle





\begin{frame}{Lyapunov method: stability criteria}
	% \framesubtitle{Parameter estimation}
	\begin{flushleft}
		
		\begin{block}{Asymptotic stability criteria}
			Autonomous dynamic system $\dot{\bo{x}} = \bo{f}(\bo{x})$ is assymptotically stable, if there exists a scalar function $V = V(\bo{x}) > 0$, whose time derivative is negative $\dot V(\bo{x}) < 0$, except $V(\bo{0}) = 0$, $\dot V(\bo{0}) = 0$.
		\end{block}
		
		\begin{block}{Marginal stability criteria}
			$\dot{\bo{x}} = \bo{f}(\bo{x})$ is stable in the sense of Lyapunov, $\exists V(\bo{x}) > 0$, $\dot V(\bo{x}) \leq 0$.
		\end{block}
		
		\begin{definition}
			Function $V(\bo{x}) > 0$ in this case is called \emph{Lyapunov function}.
		\end{definition}
		
		\bigskip
		
		This is not the only type of stability as you remember, you are invited to study criteria for other stability types on your own.
		
	\end{flushleft}
\end{frame}


\begin{frame}{Basic second-order system}
	\begin{flushleft}
		
		Let us consider the following basic system:
		
		\begin{equation}
			\ddot{\bo{q}} + \bo{K}_d \dot{\bo{q}} + \bo{K}_p\bo{q} = 0
		\end{equation}
		%
		where $\bo{K}_d > 0$ and $\bo{K}_p > 0$ are semidefinite matrices. Can we analyze its stability?
		
		\bigskip
		
		We propose Lyapunov function
		 
		 \begin{equation}
		 	V = \frac{1}{2} \dot{\bo{q}}\T  \dot{\bo{q}} + 
		 	       \frac{1}{2} \bo{q}\T \bo{K}_p\bo{q}
		 \end{equation}
		
	\end{flushleft}
\end{frame}



\begin{frame}{Basic second-order system}
	\begin{flushleft}
		
		Derivative of the Lyapunov function is:
		
		\begin{equation}
			\dot V = \dot{\bo{q}}\T  \ddot{\bo{q}} + 
			\bo{q}\T \bo{K}_p \dot{\bo{q}}
		\end{equation}
		
		But we know that $\ddot{\bo{q}} = -\bo{K}_d \dot{\bo{q}} - \bo{K}_p\bo{q}$, so:
		%
		\begin{align}
	\dot V = -\dot{\bo{q}}\T  (\bo{K}_d \dot{\bo{q}} + \bo{K}_p\bo{q}) + 
	\bo{q}\T \bo{K}_p \dot{\bo{q}} 
	\\
	\dot V = -\dot{\bo{q}}\T\bo{K}_d \dot{\bo{q}} 
		\end{align}		
	
	This means that the system is at least marginally stable: $\dot V \leq 0$, namely $\dot V = 0$ for $\dot{\bo{q}} = 0$ and $\forall \bo{q}$.
		
		\bigskip
		
		Now, we want to prove that the only trajectory in the fixed point ($\dot{\bo{q}} = 0$, $\ddot{\bo{q}} = 0$) is $\bo{q} = 0$.
		
	\end{flushleft}
\end{frame}




\begin{frame}{Basic second-order system}
	\begin{flushleft}
		
		Consider the fixed point $\dot{\bo{q}} = 0$, $\ddot{\bo{q}} = 0$:
		
		\begin{equation}
			\ddot{\bo{q}} + \bo{K}_d \dot{\bo{q}} + \bo{K}_p\bo{q} = 0 
			\ \ \ \rightarrow \ \ \ 
			\bo{K}_p\bo{q} = 0
		\end{equation}
		
		Since $\bo{K}_p > 0$, it is full rank and has a trivial null space, so $\bo{q} = 0 $ is the only trajectory in the fixed point space. So, the system is asymptotically stable by LaSalle's invariance principle.
		
	\end{flushleft}
\end{frame}





\begin{frame}{Gravity compensation}
	\begin{flushleft}
		
		Now, we can consider manipulator equations:
		
		\begin{equation}
			\bo{H} \ddot{\bo{q}} + \bo{C} \dot{\bo{q}} + \bo{g} = \tau
		\end{equation}
	
		We propose a Lyapunov function:
		 
\begin{equation}
	V = \frac{1}{2} \dot{\bo{q}}\T  \bo{H} \dot{\bo{q}} + 
	\frac{1}{2} \bo{q}\T \bo{K}_p\bo{q}
\end{equation}		
		
	\begin{align}
	\dot V = \dot{\bo{q}}\T  \bo{H}  \ddot{\bo{q}} + 
	\frac{1}{2} \dot{\bo{q}}\T  \dot{\bo{H}} \dot{\bo{q}} +
	\dot{\bo{q}}\T \bo{K}_p \bo{q} 
	\\
	\dot V = \dot{\bo{q}}\T  (\tau - \bo{C} \dot{\bo{q}} + \bo{g} ) + 
	\frac{1}{2} \dot{\bo{q}}\T  \dot{\bo{H}} \dot{\bo{q}} +
	\dot{\bo{q}}\T \bo{K}_p \bo{q} 
	\\
	\dot V = \frac{1}{2} \dot{\bo{q}}\T  ( \dot{\bo{H}} - 2 \bo{C} ) \dot{\bo{q}} +
	\dot{\bo{q}}\T  (\tau - \bo{g} ) + 
	\dot{\bo{q}}\T \bo{K}_p \bo{q} 
	\end{align}		
		
	\end{flushleft}
\end{frame}



\begin{frame}{Gravity compensation}
	\begin{flushleft}
		
		Looking at the expression: 
		
		$$\dot V = \frac{1}{2} \dot{\bo{q}}\T  ( \dot{\bo{H}} - 2 \bo{C} ) \dot{\bo{q}} +
		\dot{\bo{q}}\T  (\tau - \bo{g} ) + 
		\dot{\bo{q}}\T \bo{K}_p \bo{q} $$ 
		%
		we can observe that $ \dot{\bo{H}} - 2 \bo{C} $ is skew-symmetric, and therefore $\dot{\bo{q}}\T  ( \dot{\bo{H}} - 2 \bo{C} ) \dot{\bo{q}} = 0$
		
		We can propose the following $\tau$:
		
		\begin{equation}
			\tau = \bo{g}  - \bo{K}_d \dot{\bo{q}}  - \bo{K}_p \bo{q}
		\end{equation}		
	
		Then the Lyapunov function derivative takes form:
		
	\begin{align}
	\dot V = 
	-\dot{\bo{q}}\T  \bo{K}_d \dot{\bo{q}}
	-\dot{\bo{q}}\T  \bo{K}_p \bo{q}
	 + 
	\dot{\bo{q}}\T \bo{K}_p \bo{q} 
	\\
	\dot V = 
	-\dot{\bo{q}}\T  \bo{K}_d \dot{\bo{q}} \leq 0
	\end{align}				
					
		
	\end{flushleft}
\end{frame}




\begin{frame}{Gravity compensation}
	\begin{flushleft}
		
		So, $\dot V = 
		-\dot{\bo{q}}\T  \bo{K}_d \dot{\bo{q}} \leq 0$. More precisely $\dot V = 0$ for $\dot{\bo{q}} = 0$ and $\forall \bo{q}$, giving us marginal stability.
		
		\bigskip
		
		Let us study the fixed points $\dot{\bo{q}} = 0$ and $\ddot{\bo{q}} = 0$ for the manipulator eq.:
		
		\begin{align}
			\bo{g} = \bo{g}  - \bo{K}_p \bo{q} \\
			\bo{K}_p \bo{q} = 0 \\
			\bo{q} = 0
		\end{align}				
	
		We get that the only trajectory in the fixed space is $\bo{q} = 0$. So, the system is asymptotically stable by LaSalle's invariance principle.	
		
	\end{flushleft}
\end{frame}




\begin{frame}{Gravity compensation}
	\begin{flushleft}
		
		The control that we proposed:
		
		\begin{equation}
			\tau = \bo{g}  - \bo{K}_d \dot{\bo{q}}  - \bo{K}_p \bo{q}
		\end{equation}		
	%
	is gravity compensation and PD control. We have proven that it is  stable.
		
	\end{flushleft}
\end{frame}


\begin{frame}{Proportional control}
	\begin{flushleft}
		
		If the system has linear dissipation force (viscous force) $\bo{F} \dot{\bo{q}}$:
		
		\begin{equation}
			\bo{H} \ddot{\bo{q}} + \bo{C} \dot{\bo{q}} + \bo{F} \dot{\bo{q}} + \bo{g} = \tau
		\end{equation}
		
		...we can use proportional control without derivative component:
		
		\begin{equation}
			\tau = \bo{g}  - \bo{K}_p \bo{q}
		\end{equation}		
	
	Where the same Lyapunov function will obtain derivative:
	
	\begin{align}
		\dot V = 
		-\dot{\bo{q}}\T  \bo{F}_d \dot{\bo{q}} \leq 0
	\end{align}	
	
		With the same conclusions as before. An example of system for which we can use it is industrial robot arms with gear box reducers.
		
	\end{flushleft}
\end{frame}





\begin{frame}{Computed torque control}
	\begin{flushleft}
		
		Let us introduce  \emph{desired position} $\bo{q}^*$ and \emph{position error}  $\bo{e}$:
		
		\begin{equation}
			\bo{e} = \bo{q}^* - \bo{q}
		\end{equation}
	
		If we are lucky, the \emph{error dynamics} is evolving in accord with the following equation: 
		
		\begin{equation}
	\ddot{\bo{e}} + \bo{K}_d \dot{\bo{e}} + \bo{K}_p \bo{e} = 0
		\end{equation}		
	%
	where $\bo{K}_d > 0$ and $\bo{K}_p > 0$. This system is asymptotically stable, meaning that $\bo{e} \rightarrow 0$ and $\bo{q} \rightarrow \bo{q}^*$.
		
	\end{flushleft}
\end{frame}




\begin{frame}{Computed torque control}
	\begin{flushleft}
		
		Let us find what  $\ddot{\bo{e}}$ is:
		
		\begin{align}
			\ddot{\bo{e}} = \ddot{\bo{q}}^* - \ddot{\bo{q}} 
			\\
			\ddot{\bo{e}} = \ddot{\bo{q}}^* -  \bo{H}^{-1}(\tau - \bo{C} \dot{\bo{q}} - \bo{g})
		\end{align}
	
		Then error dynamics is:
		%
		\begin{align}
\ddot{\bo{q}}^* -  \bo{H}^{-1}(\tau - \bo{C} \dot{\bo{q}} - \bo{g})
 + \bo{K}_d \dot{\bo{e}} + \bo{K}_p \bo{e} = 0
 \\
\bo{H}\ddot{\bo{q}}^* - (\tau - \bo{C} \dot{\bo{q}} - \bo{g})
+ \bo{H}(\bo{K}_d \dot{\bo{e}} + \bo{K}_p \bo{e}) = 0 
\\
\tau = \bo{H}\ddot{\bo{q}}^* + \bo{C} \dot{\bo{q}} + \bo{g} +
\bo{H}(\bo{K}_d \dot{\bo{e}} + \bo{K}_p \bo{e})
		\end{align}				
	
	So, we found a control law that makes error dynamics stable, meaning $\bo{q} \rightarrow \bo{q}^*$.
		
	\end{flushleft}
\end{frame}




\begin{frame}{Read more}
	% \framesubtitle{Parameter estimation}
	\begin{flushleft}
		
		You can read more at \emph{ Siciliano, B., Sciavicco, L., Villani, L. and Oriolo, G., 2009. Robotics. Advanced textbooks in control and signal processing}, Chapter 8.5
		

\end{flushleft}
\end{frame}


\begin{frame}{Thank you!}
\centerline{Lecture slides are available via Moodle.}
\bigskip
\centerline{You can help improve these slides at:}
\centerline{\mygit}
\bigskip
\centerline{Check Moodle for additional links, videos, textbook suggestions.}
\bigskip

\centerline{\textcolor{black}{\qrcode[height=1.6in]{https://github.com/SergeiSa/Fundamentals-of-robotics-2022}}}

\end{frame}

\end{document}
