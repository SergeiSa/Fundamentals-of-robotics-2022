\documentclass{beamer}

\input{settings.tex}

\newcommand{\dq} {\dot{\mathbf{q}}}


\title{Compliance control, Force control}
\subtitle{Fundamentals of Robotics, Lecture 10}
\author{by Sergei Savin}
\centering
\date{\mydate}



\begin{document}
\maketitle





\begin{frame}{Operation space}
	% \framesubtitle{Parameter estimation}
	\begin{flushleft}
		
		Let us consider a task $\bo{r}_K = \bo{r}_K(\bo{q})$. We can differentiate it twice:
		
		\begin{align}
			\dot{\bo{r}}_K = \bo{J}_K \dot{\bo{q}} \\
			\ddot{\bo{r}}_K = \bo{J}_K \ddot{\bo{q}} + \dot{\bo{J}}_K \dot{\bo{q}} 
	    \end{align}
    
    	But remember, we know what robot dynamics is:
    	
        \begin{equation}
    		\bo{H} \ddot{\bo{q}} + \bo{C} \dot{\bo{q}} + \bo{g} = \tau
    	\end{equation}
    	
    	Expressing $ \ddot{\bo{q}}$ and substituting it to $\ddot{\bo{r}}_K$ we get:
    	
    	\begin{align}
    		\ddot{\bo{r}}_K = \bo{J}_K \bo{H}^{-1} (\tau - \bo{C} \dot{\bo{q}} - \bo{g}) + \dot{\bo{J}}_K \dot{\bo{q}} 
    	\end{align}
    	
		
	\end{flushleft}
\end{frame}





\begin{frame}{Operation space}
	% \framesubtitle{Parameter estimation}
	\begin{flushleft}
		
		If $\bo{J}_K$ is full rank (which happens with some robot arms) we can rewrite the last equation further:
		
		\begin{align}
			\ddot{\bo{r}}_K = \bo{J}_K \bo{H}^{-1} (\tau - \bo{C} \dot{\bo{q}} - \bo{g}) + \dot{\bo{J}}_K \dot{\bo{q}} 
			\\
			\bo{H}\bo{J}_K^{-1} \ddot{\bo{r}}_K =  (\tau - \bo{C} \dot{\bo{q}} - \bo{g}) + \bo{H}\bo{J}_K^{-1} \dot{\bo{J}}_K \dot{\bo{q}} 
		\end{align}
		
		Then we can multiply both sides by $ \bo{J}_K^{-\top}$ and define $\bo{J}_K\T \bo{f}_K = \tau $ we get:
		
		\begin{align}
	\bo{J}_K^{-\top}\bo{H}\bo{J}_K^{-1} \ddot{\bo{r}}_K
	 =  
	 \bo{J}_K^{-\top}\tau - \bo{J}_K^{-\top} \bo{C} \dot{\bo{q}} - \bo{J}_K^{-\top}\bo{g} + \bo{J}_K^{-\top} \bo{H}\bo{J}_K^{-1} \dot{\bo{J}}_K \dot{\bo{q}}
	  \\
	\bo{J}_K^{-\top}\bo{H}\bo{J}_K^{-1} \ddot{\bo{r}}_K
=  
\bo{f}_K - \bo{J}_K^{-\top} \bo{C} \dot{\bo{q}} - \bo{J}_K^{-\top}\bo{g} + \bo{J}_K^{-\top} \bo{H}\bo{J}_K^{-1} \dot{\bo{J}}_K \dot{\bo{q}}	  
		\end{align}
		
	\end{flushleft}
\end{frame}



\begin{frame}{Operation space dynamics}
	% \framesubtitle{Parameter estimation}
	\begin{flushleft}
		
		We define operation space inertial matrix $\bo{H}_K = \bo{J}_K^{-\top}\bo{H}\bo{J}_K^{-1}$:
		
		\begin{align}
			\bo{H}_K \ddot{\bo{r}}_K
			=  
			\bo{f}_K - \bo{J}_K^{-\top} \bo{C} \dot{\bo{q}} - \bo{J}_K^{-\top}\bo{g} + \bo{H}_K \dot{\bo{J}}_K \dot{\bo{q}}	  
			\\
			\bo{H}_K \ddot{\bo{r}}_K + 
			\bo{J}_K^{-\top} \bo{C} \dot{\bo{q}} - \bo{H}_K \dot{\bo{J}}_K \dot{\bo{q}}			 + 
			\bo{J}_K^{-\top}\bo{g} 
			=  
			\bo{f}_K
		\end{align}
	
		We additionally define:
		
				\begin{align}
			\bo{J}_K^{-\top} \bo{C} \dot{\bo{q}} - \bo{H}_K \dot{\bo{J}}_K \dot{\bo{q}} =  \bo{C}_K \dot{\bo{r}}_K
			\\
			\bo{J}_K^{-\top}\bo{g} 
			=  
			\bo{g} _K
		\end{align}
		
		 Which gets us to the operation-space equations:
		
		\begin{align}
			\bo{H}_K \ddot{\bo{r}}_K + \bo{C}_K \dot{\bo{r}}_K +
			\bo{g} _K
			=  
			\bo{f}_K
		\end{align}		
		
	\end{flushleft}
 \end{frame}





\begin{frame}{Operation space dynamics}
	% \framesubtitle{Parameter estimation}
	\begin{flushleft}
		
		Let us look at these equations closely:
		
		\begin{align}
			\bo{H}_K \ddot{\bo{r}}_K + \bo{C}_K \dot{\bo{r}}_K +
			\bo{g} _K
			=  
			\bo{f}_K
		\end{align}		
	
		\begin{itemize}
			\item They tell as how the task will evolve  under the influence of input forces $\bo{f}_K$
			
			\item Operation-space inertia matrix is full rank as long as $\bo{J}_K$ is full rank.
			
			\item As long as $\bo{J}_K$ is full rank, there is one-to-one correspondence between the operational space and joint space dynamics.
		\end{itemize}
	
		
	\end{flushleft}
\end{frame}



\begin{frame}{Operation space - external force}
	% \framesubtitle{Parameter estimation}
	\begin{flushleft}
		
		Assume the original dynamics includes external force $ \bo{f}_e$ with jacobian $\bo{J}_e$:
		
		\begin{equation}
			\bo{H} \ddot{\bo{q}} + \bo{C} \dot{\bo{q}} + \bo{g} = \tau + \bo{J}_e\T \bo{f}_e
		\end{equation}
		
		Then we can write the corresponding operational space dynamics:
		
		\begin{align}
	\bo{H}_K \ddot{\bo{r}}_K + \bo{C}_K \dot{\bo{r}}_K +
	\bo{g} _K
	=  
	\bo{f}_K + \bo{T}_e \bo{f}_e
		\end{align}				
	
	where $\bo{T}_e = \bo{J}_K^{-\top}\bo{J}_e\T$.
		
	\end{flushleft}
\end{frame}



\begin{frame}{Stiffness}
	% \framesubtitle{Parameter estimation}
	\begin{flushleft}
		
		If external force is elastic, proportional to the displacement in $ \bo{r}_K$, it can be re-written as:
		
		\begin{equation}
			\bo{f}_e = \bo{K}_e (\bo{r}_K^* - \bo{r}_K) = -\bo{K}_e \bo{e}_K
		\end{equation}
		
		where $\bo{e}_K = \bo{r}_K - \bo{r}_K^*$ - displacement in $ \bo{r}_K$ and $\bo{K}_e$ is a full-rank stiffness matrix.
		
		\bigskip
		
		If $\bo{r}_K^* = const$, then $\dot{\bo{e}}_K = \dot{\bo{r}}_K$ and $\ddot{\bo{e}}_K = \ddot{\bo{r}}_K$, and the dynamics assumes the form:
		%
		\begin{align}
	\bo{H}_K \ddot{\bo{e}}_K + \bo{C}_K \dot{\bo{e}}_K +
	\bo{g} _K
	=  
	\bo{f}_K -\bo{T}_e \bo{K}_e \bo{e}_K
	\\
	\bo{H}_K \ddot{\bo{e}}_K + \bo{C}_K \dot{\bo{e}}_K +
	\bo{T}_e \bo{K}_e \bo{e}_K	+
	\bo{g} _K
	=  
	\bo{f}_K
		\end{align}			
		
	\end{flushleft}
\end{frame}





\begin{frame}{Stability in operational space}
	\begin{flushleft}
		
	We propose Lyapunov function:
		 
\begin{equation}
	V = \frac{1}{2} \dot{\bo{e}}_K\T  \bo{H}_K \dot{\bo{e}}_K + 
	       \frac{1}{2} \bo{e}_K\T \bo{K}_p \bo{e}_K
\end{equation}		
		
		Let us find its time-derivative:
		
\begin{align*}
	\dot V = \dot{\bo{e}}_K\T  \bo{H}_K \ddot{\bo{e}}_K + 
				  \frac{1}{2} \dot{\bo{e}}_K\T  \dot{\bo{H}}_K \dot{\bo{e}}_K+
	               \dot{\bo{e}}_K\T \bo{K}_p \bo{e}_K
	               \\
	\dot V = \dot{\bo{e}}_K\T  \bo{f}_K - 
	\dot{\bo{e}}_K\T (\bo{C}_K \dot{\bo{e}}_K +
	\bo{T}_e \bo{K}_e \bo{e}_K	+
	\bo{g} _K) + \\
	+\frac{1}{2} \dot{\bo{e}}_K\T  \dot{\bo{H}}_K \dot{\bo{e}}_K +
	\dot{\bo{e}}_K\T \bo{K}_p \bo{e}_K             
	\\
	\dot V = \dot{\bo{e}}_K\T  \bo{f}_K  - 
	\dot{\bo{e}}_K\T \bo{T}_e \bo{K}_e \bo{e}_K	
	- \dot{\bo{e}}_K\T \bo{g} _K + 
	\dot{\bo{e}}_K\T \bo{K}_p \bo{e}_K
\end{align*}				
		
		
	\end{flushleft}
\end{frame}



\begin{frame}{Gravity compensation in operational space}
	\begin{flushleft}
		
		We can propose the following control law
		
		\begin{equation}
			\bo{f}_K = \bo{g} _K - \bo{K}_p \bo{e}_K - \bo{K}_d \dot{\bo{e}}_K + \bo{T}_e \bo{K}_e \bo{e}_K
		\end{equation}
	%
	where $\bo{K}_p$,  $\bo{K}_d$ are positive-definite matrices. Then, derivative of the Lyapunov function is:
		
	\begin{align*}
	\dot V &= \dot{\bo{e}}_K\T  
	(\bo{g} _K - \bo{K}_p \bo{e}_K + \bo{T}_e \bo{K}_e \bo{e}_K	- \bo{K}_d \dot{\bo{e}}_K)  - \\
&-\dot{\bo{e}}_K\T \bo{T}_e \bo{K}_e \bo{e}_K	
- \dot{\bo{e}}_K\T \bo{g} _K + 
\dot{\bo{e}}_K\T \bo{K}_p \bo{e}_K
\\
\dot V &= -\dot{\bo{e}}_K\T \bo{K}_d \dot{\bo{e}}_K
	\end{align*}		
		
		We can see that $\dot V \leq 0$.
		
	\end{flushleft}
\end{frame}



\begin{frame}{Gravity compensation in operational space}
	\begin{flushleft}
		
		We can consider the fixed points $\dot{\bo{e}}_K = 0$, $\ddot{\bo{e}}_K = 0$:
		
		\begin{align*}
			\bo{H}_K \ddot{\bo{e}}_K + \bo{C}_K \dot{\bo{e}}_K +
			\bo{T}_e \bo{K}_e \bo{e}_K	+
			\bo{g} _K
			&=  
			\bo{g} _K - \bo{K}_p \bo{e}_K - \bo{K}_d \dot{\bo{e}}_K + \bo{T}_e \bo{K}_e \bo{e}_K
			\\
			\bo{T}_e \bo{K}_e \bo{e}_K	+
			\bo{g} _K
			&=  
			\bo{g} _K - \bo{K}_p \bo{e}_K + \bo{T}_e \bo{K}_e \bo{e}_K
			\\
			0 &=  
			- \bo{K}_p \bo{e}_K
		\end{align*}		
	
	Since $\bo{K}_p$ is full rank, therefore $ \bo{e}_K = 0$. So, the system is asymptotically stable.
		
	\end{flushleft}
\end{frame}


\begin{frame}{Control law}
	\begin{flushleft}
		
		\begin{align}
			\bo{f}_K &= \bo{g} _K - \bo{K}_p \bo{e}_K - \bo{K}_d \dot{\bo{e}}_K + \bo{T}_e \bo{K}_e \bo{e}_K
			\\
			\tau &= \bo{J}_K\T \bo{g} _K - 
			                   \bo{J}_K\T\bo{K}_p \bo{e}_K - 
			                   \bo{J}_K\T\bo{K}_d \dot{\bo{e}}_K + 
			\bo{J}_K\T\bo{T}_e \bo{K}_e \bo{e}_K 
			\\
			\tau &= \bo{J}_K\T \textcolor{mydarkblue}{\bo{g} _K} + 
			(\bo{J}_K\T \textcolor{mygreen}{\bo{T}_e} \bo{K}_e  - \bo{J}_K\T\bo{K}_p) \bo{e}_K - 
			\bo{J}_K\T\bo{K}_d \bo{J}_e \dot{\bo{q}}
			\\
			\tau &= \bo{J}_K\T \textcolor{mydarkblue}{\bo{J}_K^{-\top}\bo{g}} + 
			(\bo{J}_K\T \textcolor{mygreen}{\bo{J}_K^{-\top}\bo{J}_e\T}  \bo{K}_e  - \bo{J}_K\T\bo{K}_p) \bo{e}_K - 
			\bo{J}_K\T\bo{K}_d \bo{J}_e \dot{\bo{q}}
			\\
			\tau &= \bo{g} + 
			(\bo{J}_e\T  \bo{K}_e  - \bo{J}_K\T\bo{K}_p) \bo{e}_K - 
			\bo{J}_K\T\bo{K}_d \bo{J}_e \dot{\bo{q}}
		\end{align}
		
		In a case when $\bo{J}_e = \bo{J}_K = \bo{J}$, we get:
		
		\begin{align}
			\tau &= \bo{g} + 
			\bo{J}\T  (\bo{K}_e  - \bo{K}_p) \bo{e}_K - 
			\bo{J}\T\bo{K}_d \bo{J} \dot{\bo{q}}
		\end{align}
		
		
		
	\end{flushleft}
\end{frame}




\begin{frame}{Control law}
	\begin{flushleft}
		
		Resulting control law:
		
		\begin{align}
			\tau &= \bo{g} + 
			\bo{J}\T  (\bo{K}_e  - \bo{K}_p) \bo{e}_K - 
			\bo{J}\T\bo{K}_d \bo{J} \dot{\bo{q}}
		\end{align}
		%
		results in a robot acting as if the contact interaction is governed by an elastic force with stiffness matrix $\bo{K} = \bo{K}_e  - \bo{K}_p$.
		
		\bigskip
		
		For a system with no natural elasticity $\bo{K}_e$ we can still achieve elastic-like behavior by choosing $\bo{K}_p$.
		
		
	\end{flushleft}
\end{frame}





\begin{frame}{Read more}
	% \framesubtitle{Parameter estimation}
	\begin{flushleft}
		
		You can read more at \emph{ Siciliano, B., Sciavicco, L., Villani, L. and Oriolo, G., 2009. Robotics. Advanced textbooks in control and signal processing}, Chapter 9.2
		

\end{flushleft}
\end{frame}


\begin{frame}{Thank you!}
\centerline{Lecture slides are available via Moodle.}
\bigskip
\centerline{You can help improve these slides at:}
\centerline{\mygit}
\bigskip
\centerline{Check Moodle for additional links, videos, textbook suggestions.}
\bigskip

\centerline{\textcolor{black}{\qrcode[height=1.6in]{https://github.com/SergeiSa/Fundamentals-of-robotics-2022}}}

\end{frame}

\end{document}
