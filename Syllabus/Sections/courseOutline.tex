

\subsection{Knowledge Areas (in terms of application)}

\begin{itemize}
    \item Robotics
    \item Automation
    \item Control
    \item Computer vision
\end{itemize}

\subsection{Course Delivery}

Two lectures and two practical sessions a week (those could be labs, seminars, practice with equipment). 

\subsection{Prerequisite courses}

\begin{itemize}
    \item Strong prerequisites: Linear Algebra, Classical Mechanics.
    \item Weak prerequisites: Calculus, Control Theory.
    \item Required background knowledge: Python (alternatively Matlab or any other language suitable to work with linear algebra-heavy problems), Ubuntu (for ROS-related practical sessions) 
\end{itemize}

\subsection{Expected Learning outcomes}

The course will provide an opportunity for participants to:

\begin{itemize}
    \item Understand how to do frame transformations on serial linkages, finding positions and orientations of the links of these mechanisms, understanding concepts in analytical and numeric forward and inverse kinematics, understanding role of jacobians and their fundamental subspaces, understand for dynamical equations are derived, how inverse dynamics is solved, what is trajectory planning.
    \item Learn how to use URDF to describe robots, how to use jacobians to solve problems relating task and joint velocities, as well as forces and joint torques, learn analytic and software-based methods for dynamics equation generation, learn collocation methods for trajectory planning.
    \item Being able to write software and other code-based solutions for robot arm motions grounded in robot kinematics and dynamics.
\end{itemize}

\subsection{Expected acquired core competencies}
 
\begin{itemize}
    \item Forward and inverse kinematics.
    \item Jacobian-based analysis.
    \item Dynamics, derivations.
\end{itemize} 

\subsection{Reference Materials}
\begin{itemize}
    \item Annotated slides
    \item Online materials
    \item Educational videos
\end{itemize}

\subsection{Computer Resources}
Students will need to run computer experiments on a laptop and/or on lab computers, as well as working with the hardware. 

\subsection{Laboratory Exercises} 
There are a series of labs and electronic handouts prepared for the course.

\subsection{Laboratory Resources}
Students will be required to use and modify a software tool written in Python which run on multiple platforms (Linux, Microsoft Windows, and Mac OS). The tool requires freely available software libraries.

\subsection{Cooperation Policy and Quotations}
We encourage intensive discussion and collaboration in this class. You should feel free to discuss all aspects of the class with classmates and work with them to complete your assignments and project report. However, if you are working together, you must provide details of your contribution and that of others.